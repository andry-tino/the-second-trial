%%%%%%%%%%%%%%%%%%%%%%%%%%%%%%%%%%%%%%%%%%%%%%%%%%%%%%%%%%%%%%%%%%%%%%%%%%%%%%%
% The 2nd Trial by Andrea Tino is licensed under CC BY-NC-SA 4.0. To view a   %
% copy of this license, visit                                                 %
% http://creativecommons.org/licenses/by-nc-sa/4.0/                           %
%%%%%%%%%%%%%%%%%%%%%%%%%%%%%%%%%%%%%%%%%%%%%%%%%%%%%%%%%%%%%%%%%%%%%%%%%%%%%%%

\documentclass{tufte-book}

\hypersetup{colorlinks} % colored hyperlinks (e.g., for onscreen viewing)

%%
% Book metadata
\title{The 2nd Trial}
\author[Andrea Tino]{Andrea Tino}
\publisher{Issue I - Year 2019}

%%
% If they're installed, use Bergamo and Chantilly from www.fontsite.com.
% They're clones of Bembo and Gill Sans, respectively.
%\IfFileExists{bergamo.sty}{\usepackage[osf]{bergamo}}{}% Bembo
%\IfFileExists{chantill.sty}{\usepackage{chantill}}{}% Gill Sans

%\usepackage{microtype}

%%
% Just some sample text
\usepackage{lipsum}

%%
% For nicely typeset tabular material
\usepackage{booktabs}

%%
% For graphics / images
\usepackage{graphicx}
\setkeys{Gin}{width=\linewidth,totalheight=\textheight,keepaspectratio}
\graphicspath{{graphics/}}

% The fancyvrb package lets us customize the formatting of verbatim
% environments.  We use a slightly smaller font.
\usepackage{fancyvrb}
\fvset{fontsize=\normalsize}

%%
% Prints argument within hanging parentheses (i.e., parentheses that take
% up no horizontal space).  Useful in tabular environments.
\newcommand{\hangp}[1]{\makebox[0pt][r]{(}#1\makebox[0pt][l]{)}}

%%
% Prints an asterisk that takes up no horizontal space.
% Useful in tabular environments.
\newcommand{\hangstar}{\makebox[0pt][l]{*}}

%%
% Prints a trailing space in a smart way.
\usepackage{xspace}

%%
% Some shortcuts for Tufte's book titles.  The lowercase commands will
% produce the initials of the book title in italics.  The all-caps commands
% will print out the full title of the book in italics.
\newcommand{\vdqi}{\textit{VDQI}\xspace}
\newcommand{\ei}{\textit{EI}\xspace}
\newcommand{\ve}{\textit{VE}\xspace}
\newcommand{\be}{\textit{BE}\xspace}
\newcommand{\VDQI}{\textit{The Visual Display of Quantitative Information}\xspace}
\newcommand{\EI}{\textit{Envisioning Information}\xspace}
\newcommand{\VE}{\textit{Visual Explanations}\xspace}
\newcommand{\BE}{\textit{Beautiful Evidence}\xspace}

\newcommand{\TL}{Tufte-\LaTeX\xspace}

% Prints the month name (e.g., January) and the year (e.g., 2008)
\newcommand{\monthyear}{%
  \ifcase\month\or January\or February\or March\or April\or May\or June\or
  July\or August\or September\or October\or November\or
  December\fi\space\number\year
}


% Prints an epigraph and speaker in sans serif, all-caps type.
\newcommand{\openepigraph}[2]{%
  %\sffamily\fontsize{14}{16}\selectfont
  \begin{fullwidth}
  \sffamily\large
  \begin{doublespace}
  \noindent\allcaps{#1}\\% epigraph
  \noindent\allcaps{#2}% author
  \end{doublespace}
  \end{fullwidth}
}

% Inserts a blank page
\newcommand{\blankpage}{\newpage\hbox{}\thispagestyle{empty}\newpage}

\usepackage{units}

% Typesets the font size, leading, and measure in the form of 10/12x26 pc.
\newcommand{\measure}[3]{#1/#2$\times$\unit[#3]{pc}}

% Macros for typesetting the documentation
\newcommand{\hlred}[1]{\textcolor{Maroon}{#1}}% prints in red
\newcommand{\hangleft}[1]{\makebox[0pt][r]{#1}}
\newcommand{\hairsp}{\hspace{1pt}}% hair space
\newcommand{\hquad}{\hskip0.5em\relax}% half quad space
\newcommand{\TODO}{\textcolor{red}{\bf TODO!}\xspace}
\newcommand{\ie}{\textit{i.\hairsp{}e.}\xspace}
\newcommand{\eg}{\textit{e.\hairsp{}g.}\xspace}
\newcommand{\na}{\quad--}% used in tables for N/A cells
\providecommand{\XeLaTeX}{X\lower.5ex\hbox{\kern-0.15em\reflectbox{E}}\kern-0.1em\LaTeX}
\newcommand{\tXeLaTeX}{\XeLaTeX\index{XeLaTeX@\protect\XeLaTeX}}
% \index{\texttt{\textbackslash xyz}@\hangleft{\texttt{\textbackslash}}\texttt{xyz}}
\newcommand{\tuftebs}{\symbol{'134}}% a backslash in tt type in OT1/T1
\newcommand{\doccmdnoindex}[2][]{\texttt{\tuftebs#2}}% command name -- adds backslash automatically (and doesn't add cmd to the index)
\newcommand{\doccmddef}[2][]{%
  \hlred{\texttt{\tuftebs#2}}\label{cmd:#2}%
  \ifthenelse{\isempty{#1}}%
    {% add the command to the index
      \index{#2 command@\protect\hangleft{\texttt{\tuftebs}}\texttt{#2}}% command name
    }%
    {% add the command and package to the index
      \index{#2 command@\protect\hangleft{\texttt{\tuftebs}}\texttt{#2} (\texttt{#1} package)}% command name
      \index{#1 package@\texttt{#1} package}\index{packages!#1@\texttt{#1}}% package name
    }%
}% command name -- adds backslash automatically
\newcommand{\doccmd}[2][]{%
  \texttt{\tuftebs#2}%
  \ifthenelse{\isempty{#1}}%
    {% add the command to the index
      \index{#2 command@\protect\hangleft{\texttt{\tuftebs}}\texttt{#2}}% command name
    }%
    {% add the command and package to the index
      \index{#2 command@\protect\hangleft{\texttt{\tuftebs}}\texttt{#2} (\texttt{#1} package)}% command name
      \index{#1 package@\texttt{#1} package}\index{packages!#1@\texttt{#1}}% package name
    }%
}% command name -- adds backslash automatically
\newcommand{\docopt}[1]{\ensuremath{\langle}\textrm{\textit{#1}}\ensuremath{\rangle}}% optional command argument
\newcommand{\docarg}[1]{\textrm{\textit{#1}}}% (required) command argument
\newenvironment{docspec}{\begin{quotation}\ttfamily\parskip0pt\parindent0pt\ignorespaces}{\end{quotation}}% command specification environment
\newcommand{\docenv}[1]{\texttt{#1}\index{#1 environment@\texttt{#1} environment}\index{environments!#1@\texttt{#1}}}% environment name
\newcommand{\docenvdef}[1]{\hlred{\texttt{#1}}\label{env:#1}\index{#1 environment@\texttt{#1} environment}\index{environments!#1@\texttt{#1}}}% environment name
\newcommand{\docpkg}[1]{\texttt{#1}\index{#1 package@\texttt{#1} package}\index{packages!#1@\texttt{#1}}}% package name
\newcommand{\doccls}[1]{\texttt{#1}}% document class name
\newcommand{\docclsopt}[1]{\texttt{#1}\index{#1 class option@\texttt{#1} class option}\index{class options!#1@\texttt{#1}}}% document class option name
\newcommand{\docclsoptdef}[1]{\hlred{\texttt{#1}}\label{clsopt:#1}\index{#1 class option@\texttt{#1} class option}\index{class options!#1@\texttt{#1}}}% document class option name defined
\newcommand{\docmsg}[2]{\bigskip\begin{fullwidth}\noindent\ttfamily#1\end{fullwidth}\medskip\par\noindent#2}
\newcommand{\docfilehook}[2]{\texttt{#1}\index{file hooks!#2}\index{#1@\texttt{#1}}}
\newcommand{\doccounter}[1]{\texttt{#1}\index{#1 counter@\texttt{#1} counter}}

% Generates the index
\usepackage{makeidx}
\makeindex

\begin{document}

% Front matter
\frontmatter

% Blank page before title
%\blankpage


% r.3 full title page
\maketitle


% v.4 copyright page
\newpage
\begin{fullwidth}
~\vfill
\thispagestyle{empty}
\setlength{\parindent}{0pt}
\setlength{\parskip}{\baselineskip}
The 2nd Trial by \thanklessauthor

\par\smallcaps{Series: \thanklesspublisher}

\par\smallcaps{github.com/andry-tino/the-second-trial}

\par The 2nd Trial by Andrea Tino is licensed under
Attribution-NonCommercial-ShareAlike 4.0 International.
To view a copy of this license, visit
\url{http://creativecommons.org/licenses/by-nc-sa/4.0/}.\index{license}

\par\textit{First printing, \monthyear}
\end{fullwidth}

% r.5 contents
\tableofcontents

\listoffigures

\listoftables

% r.7 dedication
\cleardoublepage
~\vfill
\begin{doublespace}
\noindent\fontsize{18}{22}\selectfont\itshape
\nohyphenation
Dedicated to those who appreciate \LaTeX{} 
and the work of \mbox{Edward R.~Tufte} 
and \mbox{Donald E.~Knuth}.
\end{doublespace}
\vfill
\vfill


% r.9 introduction
\cleardoublepage
\chapter*{Introduction}

This sample book discusses the design of Edward Tufte's
books\cite{Tufte2001,Tufte1990,Tufte1997,Tufte2006}
and the use of the \doccls{tufte-book} and \doccls{tufte-handout} document classes.


%%
% Start the main matter (normal chapters)
\mainmatter


\chapter[On the Use of the tufte-book Document Class]{On the Use of the \texttt{tufte-book} Document Class}
\label{ch:tufte-book}

The \TL document classes define a style similar to the
style Edward Tufte uses in his books and handouts.  Tufte's style is known
for its extensive use of sidenotes, tight integration of graphics with
text, and well-set typography.  This document aims to be at once a
demonstration of the features of the \TL document classes
and a style guide to their use.

\section{Page Layout}\label{sec:page-layout}
\subsection{Headings}\label{sec:headings}\index{headings}
This style provides \textsc{a}- and \textsc{b}-heads (that is,
\Verb|\section| and \Verb|\subsection|), demonstrated above.

If you need more than two levels of section headings, you'll have to define
them yourself at the moment; there are no pre-defined styles for anything below
a \Verb|\subsection|.  As Bringhurst points out in \textit{The Elements of
Typographic Style},\cite{Bringhurst2005} you should ``use as many levels of
headings as you need: no more, and no fewer.''

The \TL classes will emit an error if you try to use
\linebreak\Verb|\subsubsection| and smaller headings.

% let's start a new thought -- a new section
\newthought{In his later books},\cite{Tufte2006} Tufte
starts each section with a bit of vertical space, a non-indented paragraph,
and sets the first few words of the sentence in \textsc{small caps}.  To
accomplish this using this style, use the \doccmddef{newthought} command:
\begin{docspec}
  \doccmd{newthought}\{In his later books\}, Tufte starts\ldots
\end{docspec}


\section{Sidenotes}\label{sec:sidenotes}
One of the most prominent and distinctive features of this style is the
extensive use of sidenotes.  There is a wide margin to provide ample room
for sidenotes and small figures.  Any \doccmd{footnote}s will automatically
be converted to sidenotes.\footnote{This is a sidenote that was entered
using the \texttt{\textbackslash footnote} command.}  If you'd like to place ancillary
information in the margin without the sidenote mark (the superscript
number), you can use the \doccmd{marginnote} command.\marginnote{This is a
margin note.  Notice that there isn't a number preceding the note, and
there is no number in the main text where this note was written.}

The specification of the \doccmddef{sidenote} command is:
\begin{docspec}
  \doccmd{sidenote}[\docopt{number}][\docopt{offset}]\{\docarg{Sidenote text.}\}
\end{docspec}

Both the \docopt{number} and \docopt{offset} arguments are optional.  If you
provide a \docopt{number} argument, then that number will be used as the
sidenote number.  It will change of the number of the current sidenote only and
will not affect the numbering sequence of subsequent sidenotes.

Sometimes a sidenote may run over the top of other text or graphics in the
margin space.  If this happens, you can adjust the vertical position of the
sidenote by providing a dimension in the \docopt{offset} argument.  Some
examples of valid dimensions are:
\begin{docspec}
  \ttfamily 1.0in \qquad 2.54cm \qquad 254mm \qquad 6\Verb|\baselineskip|
\end{docspec}
If the dimension is positive it will push the sidenote down the page; if the
dimension is negative, it will move the sidenote up the page.

While both the \docopt{number} and \docopt{offset} arguments are optional, they
must be provided in order.  To adjust the vertical position of the sidenote
while leaving the sidenote number alone, use the following syntax:
\begin{docspec}
  \doccmd{sidenote}[][\docopt{offset}]\{\docarg{Sidenote text.}\}
\end{docspec}
The empty brackets tell the \Verb|\sidenote| command to use the default
sidenote number.

If you \emph{only} want to change the sidenote number, however, you may
completely omit the \docopt{offset} argument:
\begin{docspec}
  \doccmd{sidenote}[\docopt{number}]\{\docarg{Sidenote text.}\}
\end{docspec}

The \doccmddef{marginnote} command has a similar \docarg{offset} argument:
\begin{docspec}
  \doccmd{marginnote}[\docopt{offset}]\{\docarg{Margin note text.}\}
\end{docspec}

\section{References}
References are placed alongside their citations as sidenotes,
as well.  This can be accomplished using the normal \doccmddef{cite}
command.\sidenote{The first paragraph of this document includes a citation.}

The complete list of references may also be printed automatically by using
the \doccmddef{bibliography} command.  (See the end of this document for an
example.)  If you do not want to print a bibliography at the end of your
document, use the \doccmddef{nobibliography} command in its place.  

To enter multiple citations at one location,\cite[-3\baselineskip]{Tufte2006,Tufte1990} you can
provide a list of keys separated by commas and the same optional vertical
offset argument: \Verb|\cite{Tufte2006,Tufte1990}|.  
\begin{docspec}
  \doccmd{cite}[\docopt{offset}]\{\docarg{bibkey1,bibkey2,\ldots}\}
\end{docspec}

\section{Figures and Tables}\label{sec:figures-and-tables}
Images and graphics play an integral role in Tufte's work.
In addition to the standard \docenvdef{figure} and \docenvdef{tabular} environments,
this style provides special figure and table environments for full-width
floats.

Full page--width figures and tables may be placed in \docenvdef{figure*} or
\docenvdef{table*} environments.  To place figures or tables in the margin,
use the \docenvdef{marginfigure} or \docenvdef{margintable} environments as follows
(see figure~\ref{fig:marginfig}):

\begin{marginfigure}%
  \includegraphics[width=\linewidth]{helix}
  \caption{This is a margin figure.  The helix is defined by 
    $x = \cos(2\pi z)$, $y = \sin(2\pi z)$, and $z = [0, 2.7]$.  The figure was
    drawn using Asymptote (\url{http://asymptote.sf.net/}).}
  \label{fig:marginfig}
\end{marginfigure}

\begin{docspec}
\textbackslash begin\{marginfigure\}\\
  \qquad\textbackslash includegraphics\{helix\}\\
  \qquad\textbackslash caption\{This is a margin figure.\}\\
  \qquad\textbackslash label\{fig:marginfig\}\\
\textbackslash end\{marginfigure\}\\
\end{docspec}

The \docenv{marginfigure} and \docenv{margintable} environments accept an optional parameter \docopt{offset} that adjusts the vertical position of the figure or table.  See the ``\nameref{sec:sidenotes}'' section above for examples.  The specifications are:
\begin{docspec}
  \textbackslash{begin\{marginfigure\}[\docopt{offset}]}\\
  \qquad\ldots\\
  \textbackslash{end\{marginfigure\}}\\
  \mbox{}\\
  \textbackslash{begin\{margintable\}[\docopt{offset}]}\\
  \qquad\ldots\\
  \textbackslash{end\{margintable\}}\\
\end{docspec}

Figure~\ref{fig:fullfig} is an example of the \docenv{figure*}
environment and figure~\ref{fig:textfig} is an example of the normal
\docenv{figure} environment.

\begin{figure*}[h]
  \includegraphics[width=\linewidth]{sine.pdf}%
  \caption{This graph shows $y = \sin x$ from about $x = [-10, 10]$.
  \emph{Notice that this figure takes up the full page width.}}%
  \label{fig:fullfig}%
\end{figure*}

\begin{figure}
  \includegraphics{hilbertcurves.pdf}
%  \checkparity This is an \pageparity\ page.%
  \caption[Hilbert curves of various degrees $n$.][6pt]{Hilbert curves of various degrees $n$. \emph{Notice that this figure only takes up the main textblock width.}}
  \label{fig:textfig}
  %\zsavepos{pos:textfig}
\end{figure}

As with sidenotes and marginnotes, a caption may sometimes require vertical
adjustment. The \doccmddef{caption} command now takes a second optional
argument that enables you to do this by providing a dimension \docopt{offset}.
You may specify the caption in any one of the following forms:
\begin{docspec}
  \doccmd{caption}\{\docarg{long caption}\}\\
  \doccmd{caption}[\docarg{short caption}]\{\docarg{long caption}\}\\
  \doccmd{caption}[][\docopt{offset}]\{\docarg{long caption}\}\\
  \doccmd{caption}[\docarg{short caption}][\docopt{offset}]%
                  \{\docarg{long caption}\}
\end{docspec}
A positive \docopt{offset} will push the caption down the page. The short
caption, if provided, is what appears in the list of figures/tables, otherwise
the ``long'' caption appears there. Note that although the arguments
\docopt{short caption} and \docopt{offset} are both optional, they must be
provided in order. Thus, to specify an \docopt{offset} without specifying a
\docopt{short caption}, you must include the first set of empty brackets
\Verb|[]|, which tell \doccmd{caption} to use the default ``long'' caption. As
an example, the caption to figure~\ref{fig:textfig} above was given in the form
\begin{docspec}
  \doccmd{caption}[Hilbert curves...][6pt]\{Hilbert curves...\}
\end{docspec}

Table~\ref{tab:normaltab} shows table created with the \docpkg{booktabs}
package.  Notice the lack of vertical rules---they serve only to clutter
the table's data.

\begin{table}[ht]
  \centering
  \fontfamily{ppl}\selectfont
  \begin{tabular}{ll}
    \toprule
    Margin & Length \\
    \midrule
    Paper width & \unit[8\nicefrac{1}{2}]{inches} \\
    Paper height & \unit[11]{inches} \\
    Textblock width & \unit[6\nicefrac{1}{2}]{inches} \\
    Textblock/sidenote gutter & \unit[\nicefrac{3}{8}]{inches} \\
    Sidenote width & \unit[2]{inches} \\
    \bottomrule
  \end{tabular}
  \caption{Here are the dimensions of the various margins used in the Tufte-handout class.}
  \label{tab:normaltab}
  %\zsavepos{pos:normaltab}
\end{table}

\newthought{Occasionally} \LaTeX{} will generate an error message:\label{err:too-many-floats}
\begin{docspec}
  Error: Too many unprocessed floats
\end{docspec}
\LaTeX{} tries to place floats in the best position on the page.  Until it's
finished composing the page, however, it won't know where those positions are.
If you have a lot of floats on a page (including sidenotes, margin notes,
figures, tables, etc.), \LaTeX{} may run out of ``slots'' to keep track of them
and will generate the above error.

\LaTeX{} initially allocates 18 slots for storing floats.  To work around this
limitation, the \TL document classes provide a \doccmddef{morefloats} command
that will reserve more slots.

The first time \doccmd{morefloats} is called, it allocates an additional 34
slots.  The second time \doccmd{morefloats} is called, it allocates another 26
slots.

The \doccmd{morefloats} command may only be used two times.  Calling it a
third time will generate an error message.  (This is because we can't safely
allocate many more floats or \LaTeX{} will run out of memory.)

If, after using the \doccmd{morefloats} command twice, you continue to get the
\texttt{Too many unprocessed floats} error, there are a couple things you can
do.

The \doccmddef{FloatBarrier} command will immediately process all the floats
before typesetting more material.  Since \doccmd{FloatBarrier} will start a new
paragraph, you should place this command at the beginning or end of a
paragraph.

The \doccmddef{clearpage} command will also process the floats before
continuing, but instead of starting a new paragraph, it will start a new page.

You can also try moving your floats around a bit: move a figure or table to the
next page or reduce the number of sidenotes.  (Each sidenote actually uses
\emph{two} slots.)

After the floats have placed, \LaTeX{} will mark those slots as unused so they
are available for the next page to be composed.

\section{Captions}
You may notice that the captions are sometimes misaligned.
Due to the way \LaTeX's float mechanism works, we can't know for sure where it
decided to put a float. Therefore, the \TL document classes provide commands to
override the caption position.

\paragraph{Vertical alignment} To override the vertical alignment, use the
\doccmd{setfloatalignment} command inside the float environment.  For
example:

\begin{fullwidth}
\begin{docspec}
  \textbackslash begin\{figure\}[btp]\\
  \qquad \textbackslash includegraphics\{sinewave\}\\
  \qquad \textbackslash caption\{This is an example of a sine wave.\}\\
  \qquad \textbackslash label\{fig:sinewave\}\\
  \qquad \hlred{\textbackslash setfloatalignment\{b\}\% forces caption to be bottom-aligned}\\
  \textbackslash end\{figure\}
\end{docspec}
\end{fullwidth}

\noindent The syntax of the \doccmddef{setfloatalignment} command is:

\begin{docspec}
  \doccmd{setfloatalignment}\{\docopt{pos}\}
\end{docspec}

\noindent where \docopt{pos} can be either \texttt{b} for bottom-aligned
captions, or \texttt{t} for top-aligned captions.

\paragraph{Horizontal alignment}\label{par:overriding-horizontal}
To override the horizontal alignment, use either the \doccmd{forceversofloat}
or the \doccmd{forcerectofloat} command inside of the float environment.  For
example:

\begin{fullwidth}
\begin{docspec}
  \textbackslash begin\{figure\}[btp]\\
  \qquad \textbackslash includegraphics\{sinewave\}\\
  \qquad \textbackslash caption\{This is an example of a sine wave.\}\\
  \qquad \textbackslash label\{fig:sinewave\}\\
  \qquad \hlred{\textbackslash forceversofloat\% forces caption to be set to the left of the float}\\
  \textbackslash end\{figure\}
\end{docspec}
\end{fullwidth}

The \doccmddef{forceversofloat} command causes the algorithm to assume the
float has been placed on a verso page---that is, a page on the left side of a
two-page spread.  Conversely, the \doccmddef{forcerectofloat} command causes
the algorithm to assume the float has been placed on a recto page---that is, a
page on the right side of a two-page spread.


\section{Full-width text blocks}

In addition to the new float types, there is a \docenvdef{fullwidth}
environment that stretches across the main text block and the sidenotes
area.

\begin{Verbatim}
\begin{fullwidth}
Lorem ipsum dolor sit amet...
\end{fullwidth}
\end{Verbatim}

\begin{fullwidth}
\small\itshape\lipsum[1]
\end{fullwidth}

\section{Typography}\label{sec:typography}

\subsection{Typefaces}\label{sec:typefaces}\index{typefaces}
If the Palatino, \textsf{Helvetica}, and \texttt{Bera Mono} typefaces are installed, this style
will use them automatically.  Otherwise, we'll fall back on the Computer Modern
typefaces.

\subsection{Letterspacing}\label{sec:letterspacing}
This document class includes two new commands and some improvements on
existing commands for letterspacing.

When setting strings of \allcaps{ALL CAPS} or \smallcaps{small caps}, the
letter\-spacing---that is, the spacing between the letters---should be
increased slightly.\cite{Bringhurst2005}  The \doccmddef{allcaps} command has proper letterspacing for
strings of \allcaps{FULL CAPITAL LETTERS}, and the \doccmddef{smallcaps} command
has letterspacing for \smallcaps{small capital letters}.  These commands
will also automatically convert the case of the text to upper- or
lowercase, respectively.

The \doccmddef{textsc} command has also been redefined to include
letterspacing.  The case of the \doccmd{textsc} argument is left as is,
however.  This allows one to use both uppercase and lowercase letters:
\textsc{The Initial Letters Of The Words In This Sentence Are Capitalized.}






%%
% The back matter contains appendices, bibliographies, indices, glossaries, etc.







\backmatter

\bibliography{sample-handout}
\bibliographystyle{plainnat}


\printindex

\end{document}

