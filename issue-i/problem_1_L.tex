%%%%%%%%%%%%%%%%%%%%%%%%%%%%%%%%%%%%%%%%%%%%%%%%%%%%%%%%%%%%%%%%%%%%%%%%%%%%%%%
% The 2nd Trial by Andrea Tino is licensed under CC BY-NC-SA 4.0. To view a   %
% copy of this license, visit                                                 %
% http://creativecommons.org/licenses/by-nc-sa/4.0/                           %
%%%%%%%%%%%%%%%%%%%%%%%%%%%%%%%%%%%%%%%%%%%%%%%%%%%%%%%%%%%%%%%%%%%%%%%%%%%%%%%

\chapter[Study of functions]{Study of two functions}
\label{ch:fun}
\marginnote{Appears as \textit{first problem} in:
LI02, EA02 – Scientific; LI03 - Scientific - Applied Sciences;
LI15 - Scientific - Sport Sciences (standard session)}

This year as well we can find the classic study of function problem. We
are given 2 functions:
\begin{equation*}
    \begin{array}{cc}
        f(x) = ax^2 - x + b & g(x) = (ax + b)e^{2x - x^2} \\
    \end{array}
\end{equation*}
Before moving straight on with the questions the problem poses, let's just have
a look at $f$ and $g$ and
classify\sidenote{You need to understand who your enemy is if you
want to fight him and win!}
them. $f$ is a simple algebraic function: a second-order polynomial;
while $g$ is a more sophisticated thing: a transcendental function made up
by the combination of a first-order polynomial and an exponential term. Both
are parametric functions where, in addition to the indipendent variable
$x$, we can see parameters $a$ and $b$.

\paragraph[Problem 1]{Max and min}
\label{par:subp1_1}
\marginnote{\smallcaps{\nameref{par:subp1_1}}}
Considering $a \in \mathbb{R} \setminus \{ 0 \}$ and $b \in \mathbb{R}$, we
need to show that $g$ has a minimum and maximum (both absolute). I know at
this point you might be tempted to start calculating the first-order
derivative and the studying its sign, but you need to hold that thought
because we can solve this much more
quickly\sidenote{This is not just quicker though, it is much smarter: remember that
the problem does not ask to compute those extema, but to just prove their
existence.} without getting our hands too dirty. 

This is a maximization/minimization problem, therefore we need to
dig in our memory and find out what theorems can help us on this matter.
Weierstrass comes to mind: if we prove that $g$ is continuous inside a
closed interval, then $g$ will attain one minimum and one maximum;
however we do not have a closed interval here, we have the whole
$]-\infty, +\infty[$ so that is not a viable solution.
But we are on the right track here: \textbf{continuity is the key}.
If we can prove that $g$ is \smallcaps{continuous} and
\smallcaps{bounded}\sidenote{A function
$f(x) : \mathbb{R} \mapsto \mathbb{R}$ is bounded in
$x \in [a, b] \subset \mathbb{R}$ when there exist $m, M \in \mathbb{R}$
such that: $m \leq f(x) < M$. Note that the interval can also be
$x \in ]-\infty, +\infty[$.}
in $\mathbb{R}$, then we have
proved\sidenote{A more rigorous proof to why continuous bounded
functions attain extrema can be found at: \nameref{sec:contbound}.}
it has
absolute extrema. Think about it: continuity means that you can draw
$g$ without ever lifting your pen from the paper, the only thing
which can prevent the existence of extrema is if
$\lim_{x \to \infty} g(x) = \infty$.

Let's implement this strategy. Let's first check the domain of $g$
to see whether it is continuous. As stated earlier, this is the product
between two factors: a polynomial (always continuous in $\mathbb{R}$),
and an exponential (also continuous in $\mathbb{R}$). The argument
inside the exponential is another polynomial as well. Therefore
$g$ is defined in all $\mathbb{R}$ without any discontinuities.
We now need to prove $g$ is bounded, let's start on the left:
\begin{equation*}
    \lim_{x \to -\infty} g(x) = \lim_{x \to -\infty} (ax + b)e^{2x - x^2}
\end{equation*}
Remember that $a \neq 0$ which means tht the $x$ inside $ax + b$ is
always gonna be there, hence $\lim_{x \to -\infty} (ax + b) = -\infty$.
And of course the exponential will vanish as its argument tends
to $-\infty$: $\lim_{x \to -\infty} e^{2x - x^2} = 0$. We reach
an undecided form:
\begin{equation*}
    \lim_{x \to -\infty} (ax + b)e^{2x - x^2} =
        [-\infty \cdot 0]
\end{equation*}
Solving this indeterminate form can be done quickly by turning it into
a different one. Remember that every product $\alpha \beta$ can be
rewritten as $\frac{\alpha}{\beta^{-1}}$. What's the gain you ask?
If $\alpha \to \infty$ and $\beta \to 0$, then we have transformed a
$[\infty \cdot 0]$ form into a $[\frac{\infty}{\infty}]$.
\begin{equation*}
    \lim_{x \to -\infty} (ax + b)e^{2x - x^2} =
    \lim_{x \to -\infty} \frac{ax + b}{e^{x^2 - 2x}} =
    \left[\frac{\infty}{\infty}\right]
\end{equation*}
What good is this for? Well, now we can use De L\^{o}pital's
theorem\sidenote{When two differetiable
functions $f$ and $g$ are
such that:
\begin{equation*}
    \lim_{x \to c} \frac{f(x)}{g(x)} = \left[\frac{\infty}{\infty}\right]
\end{equation*}
Then the following holds:
\begin{equation*}
    \lim_{x \to c} \frac{f(x)}{g(x)} = \lim_{x \to c} \frac{f^\prime(x)}{g\prime(x)}
\end{equation*}
Note that $c$ can either be a number or $\infty$.}:
\begin{equation*}
    \lim_{x \to -\infty} \frac{ax + b}{e^{x^2 - 2x}} =
    \lim_{x \to -\infty} \frac{(ax + b)^\prime}{\left( e^{x^2 - 2x} \right)^\prime} =
    \lim_{x \to -\infty} \frac{a}{(2x - 2)e^{x^2 - 2x}} =
    \left[\frac{1}{\infty}\right] = 0
\end{equation*}
When we move on to take the limit of $g$ to $+\infty$,
the calculations are basically
the same and so is the result: $\lim_{x \to +\infty} g(x) = 0$.
That means the $g$ attains a maximum and a minimum because we just proved this
function is continuous and bounded.

\paragraph[Problem 2]{Parametric problem}
\label{par:subp2_1}
\marginnote{\smallcaps{\nameref{par:subp2_1}}}
We now need to find the values of $a$ and $b$ such that
$f$ and $g$'s graphs intersect at point $A \equiv (A_x, A_y) \equiv (2,1)$.
This is nothing new, we know that by putting $f$ and $g$ in a system, we can
find the values of $x$ such that the two functions will have the same value:
\begin{equation*}
    \begin{cases}
        f(A_x) = A_y\\
        g(A_x) = A_y
    \end{cases}
    \implies
    \begin{cases}
        aA_x^2 - A_x + b = A_y\\
        (aA_x + b)e^{2A_x - A_x^2} = A_y
    \end{cases}
    \implies
    \begin{cases}
        4a + b = 3\\
        2a + b = 1
    \end{cases}
\end{equation*}
By subtracting the first equation to the second, we get $2a = 2$, so:
$a = 1$ and, by substitution back in either two of the equations: $b = -1$.

\paragraph[Problem 3]{Drawing functions}
\label{par:subp3_1}
\marginnote{\smallcaps{\nameref{par:subp3_1}}}
Assuming $a = 1$ and $b = -1$, we want to draw $f$ and $g$ which become:
\begin{equation*}
    \begin{array}{cc}
        f(x) = x^2 - x -1 & g(x) = (x - 1)e^{2x - x^2} \\
    \end{array}
\end{equation*}
Let's start from $f$ which is very easy: it's a polynomial. Polynomials
are great becase we already know so many things upfront: they are
fully defined\sidenote{A function's \textit{definition} is another way
of referring to its domain.} in $\mathbb{R}$, they are not
bounded\sidenote{A function's \textit{boundedness} is a property related to
its codomain. When a function is unbounded, it means its codomain has no
restrictions, therefore the function takes values on the whole $y$ axis.},
and they diverge at $\infty$. This means
that: $g : \mathbb{R} \mapsto \mathbb{R}$, and that:
$\lim_{x \to \infty} f = \infty$. Plus, we can see $f$ is of second order,
which means it is a parabola with its concavity pointing
upwards\sidenote{A parabola $ax^2 + bx + c$ points upwards
when $a > 0$, downwards if $a < 0$.}. To draw a parabola, we only need
to find out its vertex and its intersection points with the axes.
To find the vertex you can either remember (from Geometry) its coordinates,
or use pure Calculus and treat this as a
minimization\sidenote[][0.05\textwidth]{Because the vertex
of an upward parabola happens
to also be the minimum.} problem (which means calculating the derivative
and study its sign). We are going for the second approach, but we will use
the generic equation of a parabola: $p(x) = ax^2 + bx + c$ to
find the generic coordinates of its vertex. We need to compute the first
order derivative: $p^\prime(x) = 2ax + b$ and equate it to $0$ to find
the extrema: $p^\prime(x) = 0$, which means: $2ax + b = 0$, which yields:
$x = -\frac{b}{2a}$
