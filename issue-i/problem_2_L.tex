%%%%%%%%%%%%%%%%%%%%%%%%%%%%%%%%%%%%%%%%%%%%%%%%%%%%%%%%%%%%%%%%%%%%%%%%%%%%%%%
% The 2nd Trial by Andrea Tino is licensed under CC BY-NC-SA 4.0. To view a   %
% copy of this license, visit                                                 %
% http://creativecommons.org/licenses/by-nc-sa/4.0/                           %
%%%%%%%%%%%%%%%%%%%%%%%%%%%%%%%%%%%%%%%%%%%%%%%%%%%%%%%%%%%%%%%%%%%%%%%%%%%%%%%

\chapter[Magnetism and capacitors]{Magnetic field inside a
capacitor}
\label{ch:magn}
\marginnote{Appears as \textit{second problem} in:
LI02, EA02 – Scientific; LI03 - Scientific - Applied Sciences;
LI15 - Scientific - Sport Sciences (standard session)}

This problem introduces a flat capacitor to which
a specific voltage is applied. The voltage (which is null at $t = 0$)
varies over time and it causes a magnetic field $\vec{B}$ to originate,
whose module\sidenote{Disregarding border effects.} is:
\begin{equation}\label{eq:bdef}
    B = \frac{k t}{\sqrt{\left( t^2 + a^2 \right)^3}} \cdot r
\end{equation}
Having the following
quantities\sidenote[][0.17\textwidth]{The text does not mention what MU is
assigned to $r$, but it is sensible to
infer that it should be the same as $R$ because a constraint is present
between them: $r \leq R \implies r - R \leq 0$, and since they sum together,
they must have the same MU.}:

\begin{table}[h]
\begin{center}
    \footnotesize%
    \begin{tabular}{cccl}
    \toprule
    Quantity & UM & Constraints & Description \\
    \midrule
    $\vec{B}$ & \unit{T} & & Magnetic field \\
    $R$ & \unit{m} & $R > 0$ & Capacitors plate's radius \\
    $d$ & \unit{m} & $d > 0$ & Distance between plates \\
    $r$ & \unit{m} & $r > 0 \wedge r \leq R$ & Distance from axis \\
    $k$ & & $k > 0$ & Constant \\
    $a$ & & $a > 0$ & Constant \\
    $t$ & \unit{s} & $t \geq 0$ & Time \\
    \bottomrule
    \end{tabular}
\end{center}
\caption{Most important quantities in the problem.}
\label{tab:quants}
\end{table}

\begin{figure}
    \includegraphics[width=1\linewidth]{problem_2_L_overview_diag.tikz}
    \caption[Problem representation.][0pt]{Problem representation: a 3D view and a 2D
    simplified diagram.}%
    \label{fig:capacitorovervw}%
\end{figure}

\paragraph[Problem 1]{What is the UM of $a$ and $k$?}
\label{par:subp1}
\marginnote{\smallcaps{\nameref{par:subp1}}}
We start off with a question
about \smallcaps{dimensionality}. By looking at equation \ref{eq:bdef}, we can
easily set up a dimensional equation:
\begin{equation*}
    \left[ B \right] = \left[ \frac{k t}{\sqrt{\left( t^2 + a^2 \right)^3}} r \right] \implies
    \unit{T} = \frac{[k] \cdot \unit{s} \cdot \unit{m}}{\sqrt{\left( \unit{s}^2 + [a]^2 \right)^3}}
\end{equation*}
Notice that, at denominator on the LHS, we have: $s^2 + [a]^2$. Since $[a]^2$
appears in a sum with $s^2$, it must have the same UM: $[a]^2 = s^2$.
That means $[a] = \unit{s}$, and the whole sum collapses into a single term: $s^2$:
\begin{equation*}
    \unit{T} = \frac{[k] \cdot \unit{s} \cdot \unit{m}}{\sqrt{\left( \unit{s}^2 \right)^3}} = 
    \frac{[k] \cdot \unit{s} \cdot \unit{m}}{\sqrt{\left( \unit{s}^3 \right)^2}} = 
    \frac{[k] \cdot \unit{s} \cdot \unit{m}}{\unit{s}^3} =
    \frac{[k] \cdot \unit{m}}{\unit{s}^2}
\end{equation*}
Notice how we further simplified the expression by removing the square root and removing
$\unit{s}$ terms. At this point, we have two choices: either we remember that
$T=\unit{Kg} \cdot \unit{s}^{-1} \cdot \unit{C}^{-1}$, or we try to calculate it from
Maxwell's
equations\sidenote{You can use Maxwell's 4th equation as described at: \nameref{sec:maxwell}.}.
Either way, we end up with:
\begin{equation*}
    \frac{\unit{Kg}}{\unit{s} \cdot \unit{C}} = \frac{[k] \cdot \unit{m}}{\unit{s}^2} \implies
    \frac{\unit{Kg}}{\unit{C}} = \frac{[k] \cdot \unit{m}}{\unit{s}} \implies
    \frac{\unit{Kg} \cdot \unit{s}}{\unit{C} \cdot \unit{m}} = [k]
\end{equation*}
Which means that $[k] = \unit{Kg} \cdot \unit{s} \cdot \unit{C}^{-1} \cdot \unit{m}^{-1}$.

\paragraph[Problem 2]{Why $B \neq 0$ in the capacitor even without magnets or currents?}
\label{par:subp2}
\marginnote{\smallcaps{\nameref{par:subp2}}}
The answer to this
is given by remembering that a magnetic field can originate from two possible sources:
\begin{itemize}
    \item Electrical currents.
    \item Variations of electrical fields.
\end{itemize}
If you remember Maxwell's 4th
equation\sidenote{See: \nameref{sec:maxwell}.},
then you can actually have a confirmation of these
two facts. In our problem, there are no magnets and there is no electrical current because a
capacitor opens the circuit blocking any current from flowing. However a voltage is applied,
that means an electrical field originates between the plates; and the voltage varies over
time which means the electrical field varies too, and that causes a magnetic field to
arise as per second point above.

\paragraph[Problem 3]{How do the directions of $\vec{E}$ and $\vec{B}$ relate together?}
\label{par:subp3}
\marginnote{\smallcaps{\nameref{par:subp3}}}
Here as well you should remember that inside an electromagnetic field, the electrical
field $\vec{E}$ and the magnetic field $\vec{B}$ are always perpendicular to each other.
If you didn't remember that, take Maxwell's either 3rd of 4th
equation\sidenote[][-0.1\textwidth]{The proof that $\vec{E}$ and $\vec{B}$ are perpendicular can be
found at: \nameref{sec:maxwell}.},
and you will be able to check it is true.

\paragraph[Problem 4]{Circuital current of $\vec{B}$}
\label{par:subp4}
\marginnote{\smallcaps{\nameref{par:subp4}}}
%
\begin{marginfigure}
    \includegraphics[width=1\linewidth]{problem_2_L_circuital_diag.tikz}
    \caption[Circuital current over a closed loop.]{Circuital current over a closed loop.}%
    \label{fig:circuital}%
\end{marginfigure}
%
We now want to calculate the circuital electrical current that $\vec{B}$ causes
over closed line $C$. This part is where we need to really remember the theory about
Electromagnetism. There is a precise sequence of events happening here, so let's
see what exactly happens between the capacitor's plates:
\begin{enumerate}
    \item A voltage is applied to the capacitor, causing electrical charges to accumulate on the
        plates.
    \item The presence of charges causes a uniform electrical field $\vec{E}$ to
        originate\marginnote{Maxwell's 1st law: electrical charges generate electrical fields.}.
    \item The voltage changes in time, that causes the amount of charge $Q$ on the plates to
        vary over time. This means that $\vec{E}$ changes as well over time.
    \item By virtue of Maxwell's 4th equation, we know that a changing electrical field
        causes a magnetic field $\vec{B}$ to arise.
    \item Ampere's law states that a current flows inside a closed
        line\marginnote{Also called: \textit{Amperian loop}.} permeated by a magnetic field.
        Loop $C$ is exactly that Amperian loop and $i_C$ is the so called
        \textit{displacement current} caused by $\vec{B}$.
\end{enumerate}
We want to
calculate\sidenote{At: \nameref{sec:maxwell}, I explain more in detail this process.} $i_C$!
The tool that can help us do that is Maxwell's 4th equation:
\begin{equation}
\label{eq:max4}
    \oint_C \vec{B} \cdot \,d \vec{l} = \mu_0 (i_{\text{cond}} + i_{\text{disp}})
\end{equation}
I know that is a scary equation, but I promise it's not that bad.
Equation \ref{eq:max4} states that
a magnetic field $\vec{B}$ can originate from two possible currents:
$i_{\text{cond}}$ represents the current passing through surface $S$ bounded by loop $C$;
while $i_{\text{disp}}$ is the current flowing in $C$ that originates from
the varying electrical field $\vec{E}$. So the current flowing in $C$ is the
displacement current, which means: $i_{\text{disp}} = i_C$.
So we can rewrite equation \ref{eq:max4} as:
\begin{equation}
    \label{eq:max4full}
    \oint_C \vec{B} \cdot \,d \vec{l} = \mu_0 (i_{\text{cond}} + i_C)
\end{equation}
We know that $i_C \neq 0$ because of $\vec{E}$ varying
in time. What about $i_{\text{cond}}$? That is the current that cuts through $S$. But
we are in the middle of the plates and there is no current flowing between the plates
which is flowing through $S$, that's why $i_{\text{cond}} = 0$. Thanks to these new
facts, we can now simplify equation \ref{eq:max4full} as:
\begin{equation}
    \label{eq:max4slim}
    i_C = \frac{1}{\mu_0}\oint_C \vec{B} \cdot \,d \vec{l}
\end{equation}
All right, now we really need to take care of the big elephant in the room we have
avoided looking at so far: the scary integral.
That integral is the sum of infinitely many terms: the scalar product between
the magnetic field $\vec{B}(x,y,z)$ at a certain point $(x,y,z)$ on the
loop $C$; and $\,d \vec{l}$ which the oriented piece of line $C$ at the same point.
By taking all points on $C$, calculating that scalar product for all of them, and
summing up all results, we get the value of the integral. But let's take a look at the
term $\vec{B} \cdot \,d \vec{l}$; remember that this quantity has to be evaluated for
every point on loop $C$. Being a scalar
product\sidenote{Remember that the scalar product between two vectors is a number:
    \begin{equation*}
        \vec{a} \cdot \vec{b} = a \cdot b \cdot cos\theta
    \end{equation*}
    Where $\theta$ is the angle between vectors $\vec{a}$ and $\vec{b}$.
    And it represents the length of the projection of $\vec{a}$ on $\vec{b}$.
}
we have that:
$\vec{B} \cdot \,d \vec{l} = B \,d l \cos\theta$.
What is the value of $\theta$? That is the angle between the oriented infinitesimal
loop line and the magnetic field vector. We know that the magnetic field is
perpendicular\sidenote{We discovered that at \nameref{par:subp3}.} to the
electrical field,
