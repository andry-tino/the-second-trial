%
% One column figure
%-----------------------------------------------------------
   \begin{figure}
   \centering
\tikzstyle{smallfig}=[scale=0.25]
   %
\begin{tikzpicture}[style=smallfig]
\tikzmath{
	\X1 = 0; \X2 = 8;
	\x1 = 2; \x2 = 2*\x1; \x3 = 3*\x1; 
	\xv1 = 0.5*\x1; 
	\xv2 = \x1 + 0.5*(\x2 - \x1); 
	\xv3 = \x3;
	\y1 = 1.5; \y2 = 2*\y1; \y3 = 4*\y1;
	\ya = 5;
}

\draw[thin,->] (\X1,0) -- (\X2,0) node[right] {$x$};

\draw[dotted,thin,-] (\x1,0) -- (\x1,\y3);
\draw[dotted,thin,-] (\x2,0) -- (\x2,\y3);

\draw[dashed,thin,-] (0,\y2) -- (\x1,\y2);
\draw[thin,-] (\x1,\y2) -- (\X2,\y2) node[right] {\tiny $E_1$};
\draw[thin,-] (\x1,0) -- (\x1,\y2);

\draw[dashed,thin,-] (\X1,\y1) -- (\x2,\y1);
\draw[thin,-] (\x2,\y1) -- (\X2,\y1) node[right] {\tiny $E_2$};
\draw[thin,-] (\x2,0) -- (\x2,\y1);

\node[] at (\xv1,0.5*\y1) {\tiny $-$};
\node[] at (\xv2,0.5*\y1) {\tiny $-$};
\node[] at (\xv3,0.5*\y1) {\tiny $+$};

\node[] at (\xv1,1.5*\y1) {\tiny $-$};
\node[] at (\xv2,1.5*\y1) {\tiny $+$};
\node[] at (\xv3,1.5*\y1) {\tiny $+$};

\node[blue] at (\xv1,2.5*\y1) {\tiny $+$};
\node[red] at (\xv2,2.5*\y1) {\tiny $-$};
\node[blue] at (\xv3,2.5*\y1) {\tiny $+$};

\node[blue] at (\xv1,\ya) {$\nearrow$};
\node[red] at (\xv2,\ya) {$\searrow$};
\node[blue] at (\xv3,\ya) {$\nearrow$};

\fill (\x1,0) circle (5pt) node[below] {$\beta$};
\fill (\x2,0) circle (5pt) node[below] {$0$};
\fill (\x3,0) circle (5pt) node[below] {$1$};
\end{tikzpicture}
\quad
\begin{tikzpicture}[style=smallfig]
\tikzmath{
	\X1 = 0; \X2 = 8;
	\x1 = 2; \x2 = 2*\x1; \x3 = 3*\x1; 
	\xv1 = 0.5*\x1; 
	\xv2 = \x1 + 0.5*(\x2 - \x1); 
	\xv3 = \x3;
	\y1 = 1.5; \y2 = 2*\y1; \y3 = 4*\y1;
	\ya = 5;
}

\draw[thin,->] (\X1,0) -- (\X2,0) node[right] {$x$};

\draw[dotted,thin,-] (\x1,0) -- (\x1,\y3);
\draw[dotted,thin,-] (\x2,0) -- (\x2,\y3);

\draw[dashed,thin,-] (\X1,\y1) -- (\x1,\y1);
\draw[thin,-] (\x1,\y1) -- (\X2,\y1) node[right] {\tiny $E_1$};
\draw[thin,-] (\x1,0) -- (\x1,\y1);

\draw[dashed,thin,-] (0,\y2) -- (\x2,\y2);
\draw[thin,-] (\x2,\y2) -- (\X2,\y2) node[right] {\tiny $E_2$};
\draw[thin,-] (\x2,0) -- (\x2,\y2);

\node[] at (\xv1,0.5*\y1) {\tiny $-$};
\node[] at (\xv2,0.5*\y1) {\tiny $+$};
\node[] at (\xv3,0.5*\y1) {\tiny $+$};

\node[] at (\xv1,1.5*\y1) {\tiny $-$};
\node[] at (\xv2,1.5*\y1) {\tiny $-$};
\node[] at (\xv3,1.5*\y1) {\tiny $+$};

\node[blue] at (\xv1,2.5*\y1) {\tiny $+$};
\node[red] at (\xv2,2.5*\y1) {\tiny $-$};
\node[blue] at (\xv3,2.5*\y1) {\tiny $+$};

\node[blue] at (\xv1,\ya) {$\nearrow$};
\node[red] at (\xv2,\ya) {$\searrow$};
\node[blue] at (\xv3,\ya) {$\nearrow$};

\fill (\x1,0) circle (5pt) node[below] {$0$};
\fill (\x2,0) circle (5pt) node[below] {$\beta$};
\fill (\x3,0) circle (5pt) node[below] {$1$};
\end{tikzpicture}
\quad
\begin{tikzpicture}[style=smallfig]
\tikzmath{
	\X1 = 0; \X2 = 8;
	\x1 = 2; \x2 = 2*\x1; \x3 = 3*\x1; 
	\xv1 = 0.5*\x1; 
	\xv2 = \x2; 
	\xv3 = \x3 + 0.5*(\X2 - \x3);
	\y1 = 1.5; \y2 = 2*\y1; \y3 = 4*\y1;
	\ya = 5;
}

\draw[thin,->] (\X1,0) -- (\X2,0) node[right] {$x$};

\draw[dotted,thin,-] (\x1,0) -- (\x1,\y3);
\draw[dotted,thin,-] (\x3,0) -- (\x3,\y3);

\draw[dashed,thin,-] (\X1,\y1) -- (\x1,\y1);
\draw[thin,-] (\x1,\y1) -- (\X2,\y1) node[right] {\tiny $E_1$};
\draw[thin,-] (\x1,0) -- (\x1,\y1);

\draw[dashed,thin,-] (0,\y2) -- (\x3,\y2);
\draw[thin,-] (\x3,\y2) -- (\X2,\y2) node[right] {\tiny $E_2$};
\draw[thin,-] (\x3,0) -- (\x3,\y2);

\node[] at (\xv1,0.5*\y1) {\tiny $-$};
\node[] at (\xv2,0.5*\y1) {\tiny $-$};
\node[] at (\xv3,0.5*\y1) {\tiny $+$};

\node[] at (\xv1,1.5*\y1) {\tiny $-$};
\node[] at (\xv2,1.5*\y1) {\tiny $+$};
\node[] at (\xv3,1.5*\y1) {\tiny $+$};

\node[red] at (\xv1,2.5*\y1) {\tiny $-$};
\node[red] at (\xv2,2.5*\y1) {\tiny $-$};
\node[blue] at (\xv3,2.5*\y1) {\tiny $+$};

\node[red] at (\xv1,\ya) {$\searrow$};
\node[red] at (\xv2,\ya) {$\searrow$};
\node[blue] at (\xv3,\ya) {$\nearrow$};

\fill (\x1,0) circle (5pt) node[below] {$0$};
\fill (\x2,0) circle (5pt) node[below] {$1$};
\fill (\x3,0) circle (5pt) node[below] {$\beta$};
\end{tikzpicture}
   %
   \caption{Showing the how the constraints on $f$ define its shape
   in $[0,1] \times [0,1]$. The first picture shows the regione where $f$
   can lay, the middle one shows tangent in $x=0$ and the condition
   of crossing in $(1,0)$. The rightmost picture shows possible
   shapes of $f$: note how the concavity cannot be pointing upwards.}
   \label{fig:fshape}
   \end{figure}
%-----------------------------------------------------------
%