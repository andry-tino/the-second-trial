%                                                                 aa.dem
% AA vers. 8.2, LaTeX class for Astronomy & Astrophysics
% demonstration file
%                                                       (c) EDP Sciences
%-----------------------------------------------------------------------
%
%\documentclass[referee]{aa} % for a referee version
%\documentclass[onecolumn]{aa} % for a paper on 1 column  
%\documentclass[longauth]{aa} % for the long lists of affiliations 
%\documentclass[rnote]{aa} % for the research notes
%\documentclass[letter]{aa} % for the letters 
%\documentclass[bibyear]{aa} % if the references are not structured 
% according to the author-year natbib style

%
\documentclass{layout}  

%
%%%%%%%%%%%%%%%%%%%%%%%%%%%%%%%%%%%%%%%%
\usepackage{txfonts}
\usepackage{amsmath,amssymb}
\usepackage{graphicx}
\usepackage{tikz}
%%%%%%%%%%%%%%%%%%%%%%%%%%%%%%%%%%%%%%%%
\usepackage{hyperref}
% To add links in your PDF file, use the package "hyperref"
% with options according to your LaTeX or PDFLaTeX drivers.
%
\begin{document} 


   \title{Italian Esame di Maturit\'a - Year 2018}

   \subtitle{Regular session\thanks{
                       Indirizzi: LI02, EA02 - Scientifico; LI03 - Scientifico opzione Scienze Applicate;
                       LI15 - Scientifico sezione ad Indirizzo Sportivo.} 
               (sessione ordinaria)}

   \author{A. Tino%\inst{1}
          %\inst{1}\fnmsep\thanks{Just to show the usage
          %of the elements in the author field}
          }

   \institute{Software Engineer and Mathematician,
              Copenhagen, Denmark\\
              \email{andry.tino@gmail.com}
         %\and
             %University of Alexandria, Department of Geography, ...\\
             %\email{c.ptolemy@hipparch.uheaven.space}
             %\thanks{The university of heaven temporarily does not
                     %accept e-mails}
             }

   \date{Started: September 15, 1996; Published: March 16, 1997}

% \abstract{}{}{}{}{} 
% 5 {} token are mandatory
 
  \abstract
  % context heading (optional)
  % {} leave it empty if necessary  
   {Solving the Italian exam for year 2018.}
  % aims heading (mandatory)
   {Learning the best and fastest techniques to get to solutions.}
  % methods heading (mandatory)
   {Pen and paper approach will be used to solve the problems. At the end of each section, for
    the sake of learning and knowledge, alternative processes will be shown by using language R
    get to solutions using the computer.}
  % results heading (mandatory)
   {Solutions are found.}
  % conclusions heading (optional), leave it empty if necessary 
   {}
   \keywords{mathematics --
                italy --
                maturity --
                2018 --
                high-school --
                exam --
                solutions
               }

   \maketitle
%
%________________________________________________________________

\section{Introduction}

   In the \emph{nucleated instability\/} (also called core
   instability) hypothesis of giant planet
   formation, a critical mass for static core  envelope
   protoplanets has been found. Mizuno (\cite{mizuno}) determined
   the critical mass of the core to be about $12 \,M_\oplus$
   ($M_\oplus=5.975 \times 10^{27}\,\mathrm{g}$ is the Earth mass), which
   is independent of the outer boundary
   conditions and therefore independent of the location in the
   solar nebula. This critical value for the core mass corresponds
   closely to the cores of today's giant planets.

   Although no hydrodynamical study has been available many workers
   conjectured that a collapse or rapid contraction will ensue
   after accumulating the critical mass. The main motivation for
   this article
   is to investigate the stability of the static envelope at the
   critical mass. With this aim the local, linear stability of static
   radiative gas  spheres is investigated on the basis of Baker's
   (\cite{baker}) standard one-zone model. 

   Phenomena similar to the ones described above for giant planet
   formation have been found in hydrodynamical models concerning
   star formation where protostellar cores explode
   (Tscharnuter \cite{tscharnuter}, Balluch \cite{balluch}),
   whereas earlier studies found quasi-steady collapse flows. The
   similarities in the (micro)physics, i.e., constitutive relations of
   protostellar cores and protogiant planets serve as a further
   motivation for this study.

%__________________________________________________________________

\section{Baker's standard one-zone model}

%                                     Two column figure (place early!)
%______________________________________________ Gamma_1 (lg rho, lg e)
   \begin{figure*}
   \centering
   %%%\includegraphics{empty.eps}
   %%%\includegraphics{empty.eps}
   %%%\includegraphics{empty.eps}
   \caption{Adiabatic exponent $\Gamma_1$.
               $\Gamma_1$ is plotted as a function of
               $\lg$ internal energy $\mathrm{[erg\,g^{-1}]}$ and $\lg$
               density $\mathrm{[g\,cm^{-3}]}$.}
              \label{FigGam}%
    \end{figure*}
%
   In this section the one-zone model of Ba
   originally used to study the Cephe{\"{\i}}d pulsation mechanism, will
   be briefly reviewed. The resulting stability criteria will be
   rewritten in terms of local state variables, local timescales and
   constitu

\begin{theorem}[Some]
\begin{proof}
dfdfdf
$\square$
\end{proof}
\end{theorem}

   Baker (\cite{baker}) investigates the stability of thin layers in
   self-gravitating,
   spherical gas clouds with the following properties:
   \begin{itemize}
      \item hydrostatic equilibrium,
      \item thermal equilibrium,
      \item energy transport by grey radiation diffusion.
   \end{itemize}
   For the one-zone-model Baker obtains necessary conditions
   for dynamical, secular and vibrational (or pulsational)
   stability (Eqs.\ (34a,\,b,\,c) in Baker \cite{baker}). Using Baker's
   notation:
   \[
      \begin{array}{lp{0.8\linewidth}}
         M_{r}  & mass internal to the radius $r$     \\
         m               & mass of the zone                    \\
         r_0             & unperturbed zone radius             \\
         \rho_0          & unperturbed density in the zone     \\
         T_0             & unperturbed temperature in the zone \\
         L_{r0}          & unperturbed luminosity              \\
         E_{\mathrm{th}} & thermal energy of the zone
      \end{array}
   \]
\noindent
   and with the definitions of the \emph{local cooling time\/}
   (see Fig.~\ref{FigGam})
   \begin{equation}
      \tau_{\mathrm{co}} = \frac{E_{\mathrm{th}}}{L_{r0}} \,,
   \end{equation}
   and the \emph{local free-fall time}
   \begin{equation}
      \tau_{\mathrm{ff}} =
         \sqrt{ \frac{3 \pi}{32 G} \frac{4\pi r_0^3}{3 M_{\mathrm{r}}}
}\,,
   \end{equation}
   Baker's $K$ and $\sigma_0$ have the following form:
   \begin{eqnarray}
      \sigma_0 & = & \frac{\pi}{\sqrt{8}}
                     \frac{1}{ \tau_{\mathrm{ff}}} \\
      K        & = & \frac{\sqrt{32}}{\pi} \frac{1}{\delta}
                        \frac{ \tau_{\mathrm{ff}} }
                             { \tau_{\mathrm{co}} }\,;
   \end{eqnarray}
   where $ E_{\mathrm{th}} \approx m (P_0/{\rho_0})$ has been used and
   \begin{equation}
   \begin{array}{l}
      \delta = - \left(
                    \frac{ \partial \ln \rho }{ \partial \ln T }
                 \right)_P \\
      e=mc^2
   \end{array}
   \end{equation}
   is a thermodynamical quantity which is of order $1$ and equal to $1$
   for nonreacting mixtures of classical perfect gases. The physical
   meaning of $ \sigma_0 $ and $K$ is clearly visible in the equations
   above. $\sigma_0$ represents a frequency of the order one per
   free-fall time. $K$ is proportional to the ratio of the free-fall
   time and the cooling time. Substituting into Baker's criteria, using
   thermodynamic identities and definitions of thermodynamic quantities,
   \begin{displaymath}
      \Gamma_1      = \left( \frac{ \partial \ln P}{ \partial\ln \rho}
                           \right)_{S}    \, , \;
      \chi^{}_\rho  = \left( \frac{ \partial \ln P}{ \partial\ln \rho}
                           \right)_{T}    \, , \;
      \kappa^{}_{P} = \left( \frac{ \partial \ln \kappa}{ \partial\ln P}
                           \right)_{T}
   \end{displaymath}
   \begin{displaymath}
      \nabla_{\mathrm{ad}} = \left( \frac{ \partial \ln T}
                             { \partial\ln P} \right)_{S} \, , \;
      \chi^{}_T       = \left( \frac{ \partial \ln P}
                             { \partial\ln T} \right)_{\rho} \, , \;
      \kappa^{}_{T}   = \left( \frac{ \partial \ln \kappa}
                             { \partial\ln T} \right)_{T}
   \end{displaymath}
   one obtains, after some pages of algebra, the conditions for
   \emph{stability\/} given
   below:
   \begin{eqnarray}
      \frac{\pi^2}{8} \frac{1}{\tau_{\mathrm{ff}}^2}
                ( 3 \Gamma_1 - 4 )
         & > & 0 \label{ZSDynSta} \\
      \frac{\pi^2}{\tau_{\mathrm{co}}
                   \tau_{\mathrm{ff}}^2}
                   \Gamma_1 \nabla_{\mathrm{ad}}
                   \left[ \frac{ 1- 3/4 \chi^{}_\rho }{ \chi^{}_T }
                          ( \kappa^{}_T - 4 )
                        + \kappa^{}_P + 1
                   \right]
        & > & 0 \label{ZSSecSta} \\
     \frac{\pi^2}{4} \frac{3}{\tau_{ \mathrm{co} }
                              \tau_{ \mathrm{ff} }^2
                             }
         \Gamma_1^2 \, \nabla_{\mathrm{ad}} \left[
                                   4 \nabla_{\mathrm{ad}}
                                   - ( \nabla_{\mathrm{ad}} \kappa^{}_T
                                     + \kappa^{}_P
                                     )
                                   - \frac{4}{3 \Gamma_1}
                                \right]
        & > & 0   \label{ZSVibSta}
   \end{eqnarray}
%
   For a physical discussion of the stability criteria see Baker
   (\cite{baker}) or Cox (\cite{cox}).

   We observe that these criteria for dynamical, secular and
   vibrational stability, respectively, can be factorized into
   \begin{enumerate}
      \item a factor containing local timescales only,
      \item a factor containing only constitutive relations and
         their derivatives.
   \end{enumerate}
   The first factors, depending on only timescales, are positive
   by definition. The signs of the left hand sides of the
   inequalities~(\ref{ZSDynSta}), (\ref{ZSSecSta}) and (\ref{ZSVibSta})
   therefore depend exclusively on the second factors containing
   the constitutive relations. Since they depend only
   on state variables, the stability criteria themselves are \emph{
   functions of the thermodynamic state in the local zone}. The
   one-zone stability can therefore be determined
   from a simple equation of state, given for example, as a function
   of density and
   temperature. Once the microphysics, i.e.\ the thermodynamics
   and opacities (see Table~\ref{KapSou}), are specified (in practice
   by specifying a chemical composition) the one-zone stability can
   be inferred if the thermodynamic state is specified.
   The zone -- or in
   other words the layer -- will be stable or unstable in
   whatever object it is imbedded as long as it satisfies the
   one-zone-model assumptions. Only the specific growth rates
   (depending upon the time scales) will be different for layers
   in different objects.

%__________________________________________________ One column table
   \begin{table}
      \caption[]{Opacity sources.}
         \label{KapSou}
     $$ 
         \begin{array}{p{0.5\linewidth}l}
            \hline
            \noalign{\smallskip}
            Source      &  T / {[\mathrm{K}]} \\
            \noalign{\smallskip}
            \hline
            \noalign{\smallskip}
            Yorke 1979, Yorke 1980a & \leq 1700^{\mathrm{a}}     \\
%           Yorke 1979, Yorke 1980a & \leq 1700             \\
            Kr\"ugel 1971           & 1700 \leq T \leq 5000 \\
            Cox \& Stewart 1969     & 5000 \leq             \\
            \noalign{\smallskip}
            \hline
         \end{array}
     $$ 
   \end{table}
%
   We will now write down the sign (and therefore stability)
   determining parts of the left-hand sides of the inequalities
   (\ref{ZSDynSta}), (\ref{ZSSecSta}) and (\ref{ZSVibSta}) and thereby
   obtain \emph{stability equations of state}.

   The sign determining part of inequality~(\ref{ZSDynSta}) is
   $3\Gamma_1 - 4$ and it reduces to the
   criterion for dynamical stability
   \begin{equation}
     \Gamma_1 > \frac{4}{3}\,\cdot
   \end{equation}
   Stability of the thermodynamical equilibrium demands
   \begin{equation}
      \chi^{}_\rho > 0, \;\;  c_v > 0\, ,
   \end{equation}
   and
   \begin{equation}
      \chi^{}_T > 0
   \end{equation}
   holds for a wide range of physical situations.
   With
   \begin{eqnarray}
      \Gamma_3 - 1 = \frac{P}{\rho T} \frac{\chi^{}_T}{c_v}&>&0\\
      \Gamma_1     = \chi_\rho^{} + \chi_T^{} (\Gamma_3 -1)&>&0\\
      \nabla_{\mathrm{ad}}  = \frac{\Gamma_3 - 1}{\Gamma_1}         &>&0
   \end{eqnarray}
   we find the sign determining terms in inequalities~(\ref{ZSSecSta})
   and (\ref{ZSVibSta}) respectively and obtain the following form
   of the criteria for dynamical, secular and vibrational
   \emph{stability}, respectively:
   \begin{eqnarray}
      3 \Gamma_1 - 4 =: S_{\mathrm{dyn}}      > & 0 & \label{DynSta}  \\
%
      \frac{ 1- 3/4 \chi^{}_\rho }{ \chi^{}_T } ( \kappa^{}_T - 4 )
         + \kappa^{}_P + 1 =: S_{\mathrm{sec}} > & 0 & \label{SecSta} \\
%
      4 \nabla_{\mathrm{ad}} - (\nabla_{\mathrm{ad}} \kappa^{}_T
                             + \kappa^{}_P)
                             - \frac{4}{3 \Gamma_1} =: S_{\mathrm{vib}}
                                      > & 0\,.& \label{VibSta}
   \end{eqnarray}
   The constitutive relations are to be evaluated for the
   unperturbed thermodynamic state (say $(\rho_0, T_0)$) of the zone.
   We see that the one-zone stability of the layer depends only on
   the constitutive relations $\Gamma_1$,
   $\nabla_{\mathrm{ad}}$, $\chi_T^{},\,\chi_\rho^{}$,
   $\kappa_P^{},\,\kappa_T^{}$.
   These depend only on the unperturbed
   thermodynamical state of the layer. Therefore the above relations
   define the one-zone-stability equations of state
   $S_{\mathrm{dyn}},\,S_{\mathrm{sec}}$
   and $S_{\mathrm{vib}}$. See Fig.~\ref{FigVibStab} for a picture of
   $S_{\mathrm{vib}}$. Regions of secular instability are
   listed in Table~1.

%
%                                                One column figure
%----------------------------------------------------------- S_vib
   \begin{figure}
   \centering
   %%%\includegraphics[width=3cm]{empty.eps}
      \caption{Vibrational stability equation of state
               $S_{\mathrm{vib}}(\lg e, \lg \rho)$.
               $>0$ means vibrational stability.
              }
         \label{FigVibStab}
   \end{figure}
%
%______________________________________________________________

\section{Conclusions}

   \begin{enumerate}
      \item The conditions for the stability of static, radiative
         layers in gas spheres, as described by Baker's (\cite{baker})
         standard one-zone model, can be expressed as stability
         equations of state. These stability equations of state depend
         only on the local thermodynamic state of the layer.
      \item If the constitutive relations -- equations of state and
         Rosseland mean opacities -- are specified, the stability
         equations of state can be evaluated without specifying
         properties of the layer.
      \item For solar composition gas the $\kappa$-mechanism is
         working in the regions of the ice and dust features
         in the opacities, the $\mathrm{H}_2$ dissociation and the
         combined H, first He ionization zone, as
         indicated by vibrational instability. These regions
         of instability are much larger in extent and degree of
         instability than the second He ionization zone
         that drives the Cephe{\"\i}d pulsations.
   \end{enumerate}

\begin{acknowledgements}
      Part of this work was supported by the German
      \emph{Deut\-sche For\-schungs\-ge\-mein\-schaft, DFG\/} project
      number Ts~17/2--1.
\end{acknowledgements}


%-------------------------------------------------------------------

\begin{thebibliography}{}

  \bibitem[1966]{baker} Baker, N. 1966,
      in Stellar Evolution,
      ed.\ R. F. Stein,\& A. G. W. Cameron
      (Plenum, New York) 333

   \bibitem[1988]{balluch} Balluch, M. 1988,
      A\&A, 200, 58

   \bibitem[1980]{cox} Cox, J. P. 1980,
      Theory of Stellar Pulsation
      (Princeton University Press, Princeton) 165

   \bibitem[1969]{cox69} Cox, A. N.,\& Stewart, J. N. 1969,
      Academia Nauk, Scientific Information 15, 1

   \bibitem[1980]{mizuno} Mizuno H. 1980,
      Prog. Theor. Phys., 64, 544
   
   \bibitem[1987]{tscharnuter} Tscharnuter W. M. 1987,
      A\&A, 188, 55
  
   \bibitem[1992]{terlevich} Terlevich, R. 1992, in ASP Conf. Ser. 31, 
      Relationships between Active Galactic Nuclei and Starburst Galaxies, 
      ed. A. V. Filippenko, 13

   \bibitem[1980a]{yorke80a} Yorke, H. W. 1980a,
      A\&A, 86, 286

   \bibitem[1997]{zheng} Zheng, W., Davidsen, A. F., Tytler, D. \& Kriss, G. A.
      1997, preprint
\end{thebibliography}

\end{document}
