%                                                                 aa.dem
% AA vers. 8.2, LaTeX class for Astronomy & Astrophysics
% demonstration file
%                                                       (c) EDP Sciences
%-----------------------------------------------------------------------
%
%\documentclass[referee]{aa} % for a referee version
%\documentclass[onecolumn]{aa} % for a paper on 1 column  
%\documentclass[longauth]{aa} % for the long lists of affiliations 
%\documentclass[rnote]{aa} % for the research notes
%\documentclass[letter]{aa} % for the letters 
%\documentclass[bibyear]{aa} % if the references are not structured 
% according to the author-year natbib style

%
\documentclass{layout}  

%
%%%%%%%%%%%%%%%%%%%%%%%%%%%%%%%%%%%%%%%%
\usepackage{txfonts}
\usepackage{amsmath,amssymb}
\usepackage{graphicx}

\usepackage{tikz}

\usetikzlibrary{plotmarks}
\usetikzlibrary{patterns}
\usetikzlibrary{decorations.markings}
\usetikzlibrary{math}
%%%%%%%%%%%%%%%%%%%%%%%%%%%%%%%%%%%%%%%%
\usepackage{hyperref}
% To add links in your PDF file, use the package "hyperref"
% with options according to your LaTeX or PDFLaTeX drivers.
%
\begin{document} 


   \title{Solutions of \emph{Esame di Maturit\'a} - Year 2018}

   \subtitle{Regular session\thanks{
                       Indirizzi: LI02, EA02 - Scientifico; LI03 - Scientifico opzione Scienze Applicate;
                       LI15 - Scientifico sezione ad Indirizzo Sportivo.} 
               (sessione ordinaria)}

   \author{A. Tino%\inst{1}
          %\inst{1}\fnmsep\thanks{Just to show the usage
          %of the elements in the author field}
          }

   \institute{Software Engineer and Mathematician,
              Copenhagen, Denmark\\
              \email{andry.tino@gmail.com}
         %\and
             %University of Alexandria, Department of Geography, ...\\
             %\email{c.ptolemy@hipparch.uheaven.space}
             %\thanks{The university of heaven temporarily does not
                     %accept e-mails}
             }

   \date{Started: September 15, 1996; Published: March 16, 1997}

% \abstract{}{}{}{}{} 
% 5 {} token are mandatory
 
  \abstract
  % context heading (optional)
  % {} leave it empty if necessary  
   {Analysis and resolution of the problems proposed in the Italian Mathematics national exam for year 2018.}
  % aims heading (mandatory)
   {Learning the best and fastest techniques to get to the solutions to the proposed problems.}
  % methods heading (mandatory)
   {Pen and paper approach will be used to solve the problems by employing common mathematical 
    techniques accessible to high school students. At the end of each section, for
    the sake of tutoring and learning, alternative processes will be shown by using language R to
    get to the solutions by using computer-aided techniques.}
  % results heading (mandatory)
   {The solutions are found and they are confirmed by computational calculus.}
  % conclusions heading (optional), leave it empty if necessary 
   {}
   \keywords{mathematics --
                italy --
                maturity --
                2018 --
                high-school --
                exam --
                solutions
               }

   \maketitle

%
%________________________________________________________________
% Chapter
\section{Introduction}

The first problem, as usual, offers an applicative context very common and important
nowadays in the field of \emph{Automation Factory}: here mathematical techniques
can be used to automate certain processes. The problem also integrates elements of
Probability Theory to provide a full real case analysis around process failure and how
to efficiently fix issues in a production line. The mathematical skills required to solve
this problem revolve around basic calculus, function symmetry, polynomials and
basic concepts of probability.

The second problem is more theoretical without any applicative scenarios. The
mathematical skills required to solve it include polynomials and basic calculus.

The questionnaire spans across different themes ranging from basic equations,
simple integration, Geometry, Calculus, Probability and differential equations.

The final assessment on the overall exam sets its difficulty to an average level.

%
%________________________________________________________________
% Chapter
\section{First problem}
\label{sec:p1}

A single tile is decorated with a machine which is programmed using function 
$f:[0,1] \mapsto [0,1]$. The problem \cite{exam1} claims the final graph $\Lambda$ is
a closed curve, we want to prove it first (just for fun, since this is not required).

Before doing so, we want to find the expressions of $f_1$, $f_2$ and $f_3$: respectively
the symmetrical copies of $f$ on the vertical axis, the horizontal axis and the center 
of coordinates: $O \equiv (0,0)$.
%
% Figure
%-----------------------------------------------------------
%
% One column figure
%-----------------------------------------------------------
   \begin{figure}
   \centering
\tikzstyle{smallfig}=[scale=0.25]
   %
\begin{tikzpicture}[style=smallfig]
\fill [lightgray!20] (0, -3) rectangle (4, 5);
\fill [lightgray!40] (-4, -3) rectangle (0, 5);

\draw[thin,->] (-4,0) -- (4,0) node[right] {$\hat{x}$};
\draw[thick,->,red] (0,-3) -- (0,5) node[above] {$\hat{y}$};

\fill (2,2) circle (4pt) node[above] {$A$};
\draw[dashed,thin,-] (0,2.0) -- (2,2.0);
\fill (0,2) circle (4pt) node[above left] {$f(x)$};
\draw[dashed,thin,-] (2,0) node[below] {$x$} -- (2,2);
\fill (2,0) circle (4pt);
\fill (-2,2) circle (4pt) node[below left] {$A_1$};
\draw[dashed,thin,-] (0,2.0) -- (-2,2.0);
\draw[dashed,thin,-] (-2,0) node[below] {$-x$} -- (-2,2);
\fill (-2,0) circle (4pt);
\end{tikzpicture}
\quad
\begin{tikzpicture}[style=smallfig]
\fill [lightgray!20] (0, 0) rectangle (4, 4);
\fill [lightgray!40] (-4, -4) rectangle (0, 0);

\draw[thin,->] (-4,0) -- (4,0) node[right] {$\hat{x}$};
\draw[thin,->] (0,-4) -- (0,4) node[above] {$\hat{y}$};

\fill (2,2) circle (4pt) node[above] {$A$};
\draw[dashed,thin,-] (0,2.0) -- (2,2.0);
\fill (0,2) circle (4pt) node[above left] {$f(x)$};
\draw[dashed,thin,-] (2,0) node[below] {$x$} -- (2,2);
\fill (2,0) circle (4pt);
\fill (-2,-2) circle (4pt) node[below] {$A_2$};
\draw[dashed,thin,-] (0,-2) node[below right] {$-f(x)$} -- (-2,-2);
\draw[dashed,thin,-] (-2,0) node[above] {$-x$} -- (-2,-2);
\fill (0,-2) circle (4pt);
\draw[dashed,thin,-] (2,2) -- (-2,-2);
\fill (-2,0) circle (4pt);

\node[mark size=3pt,color=red] at (0,0) {\pgfuseplotmark{triangle*}};
\end{tikzpicture}
\quad
\begin{tikzpicture}[style=smallfig]
\fill [lightgray!20] (-3, 0) rectangle (5, 4);
\fill [lightgray!40] (-3, -4) rectangle (5, 0);

\draw[thick,->,red] (-3,0) -- (5,0) node[right] {$\hat{x}$};
\draw[thin,->] (0,-4) -- (0,4) node[above] {$\hat{y}$};

\fill (2,2) circle (4pt) node[above] {$A$};
\draw[dashed,thin,-] (0,2.0) -- (2,2.0);
\fill (0,2) circle (4pt) node[above left] {$f(x)$};
\draw[dashed,thin,-] (2,0) node[below right] {$x$} -- (2,2);
\fill (2,0) circle (4pt);
\fill (2,-2) circle (4pt) node[below] {$A_3$};
\draw[dashed,thin,-] (2,0) -- (2,-2);
\draw[dashed,thin,-] (0,-2) node[left] {$-f(x)$} -- (2,-2);
\fill (0,-2) circle (4pt);
\end{tikzpicture}
   %
   \caption{The three symmetrical copies of $f$. From left to
    right: vertical, center and horizontal symmetries.}
   \label{fig:symms}
   \end{figure}
%-----------------------------------------------------------
%
%-----------------------------------------------------------
%
Figure \ref{fig:symms} shows this simple calculation.

\begin{proposition}[$\Lambda$ is a closed curve]
\label{lem:lclosed}
Let $f:[0,a] \mapsto [0,a]$, with $a > 0$, be a continuous function also satisfying:
\begin{equation}
\label{eq:fconds}
f(0)=a \wedge f(a)=0 \wedge 0 \leq f(x) \leq a, \forall x \in [0,a]
\end{equation}
Then graph $\Lambda$, obtained by adjoining together $\Gamma$ ($f$'s graph) and
its vertically, horizontally and centered symmetries, is a closed curve.
\begin{proof}
Since $\Lambda$ is constructed using $\Gamma$, given that symmentry is an 
isometric\footnote{An isometry is a transformation that does not change distances and
preserves shapes.} 
transformation and that $f$ is continuous (therefore $\Gamma$ is a continuous curve
with no interruptions), then the single thing to prove is that the points were
the copies of $\Gamma$ connect with each other are continuous. We have 4
copies of $\Gamma$:
\begin{equation*}
\begin{array}{c|c|c|c}
\text{\# Quarter} & \text{\# Ranges} & \text{Graph} & \text{\# Function} \\
\hline
\text{I (NE)} & [0,a] \times [0,a] & \Gamma & f(x) \\
\text{II (NW)} & [-a,0] \times [0,a] & \Gamma_1 & f_1 = f(-x) \\
\text{III (SW)} & [-a,0] \times [-a,0] & \Gamma_2 & f_2 = -f(-x) \\
\text{IV (SE)} & [0,a] \times [-a,0] & \Gamma_3 & f_3 = -f(x) \\
\end{array}
\end{equation*}
The last column of the table shows the definitions of the symmetrical copies of $f$.
Now we can check all 4 connection points:
\begin{equation}\label{eq:symms}
\begin{array}{l|c}
f(0) = f_1(0) \implies f(0) = f(0) & ^{\Gamma}/_{\Gamma_1} \\
f_1(-a) = f_2(-a) \implies f(a) = -f(a) \implies 0 = 0 & ^{\Gamma_1}/_{\Gamma_2} \\
f_2(0) = f_3(0) \implies -f(0) = -f(0) & ^{\Gamma_2}/_{\Gamma_3} \\
f(a) = f_3(a) \implies f(a) = -f(a) \implies 0 = 0 & ^{\Gamma_3}/_{\Gamma} \\
\end{array}
\end{equation}
And that proves the thesis.
$\square$
\end{proof}
\end{proposition}

Proposition \ref{lem:lclosed} takes a generic case, but if we consider $a=1$, we
successfully get back to our case \cite{exam1}.\\

The same problem later requires to find the definition of $f$ and its symetrical
copies $f_k$ ($k = 1 \dots 3$). This task is very simple because $f$ is a line
crossing points $(1,0)$ and $(0,1)$, therefore:
\begin{equation*}
f: \frac{x-x_1}{y-y_1} = \frac{x_2-x_1}{y_2-y_1} \wedge (x_1,y_1) = (1,0), 
(x_2,y_2) = (0,1)
\end{equation*}
Which quickly leads to: $y = 1-x$, therefore: $f(x) = 1-x$. The expressions of the
copies of $f$ were found in equation \ref{eq:symms}, so by substituting $f(x)$
in $f_1$, $f_2$ and $f_3$, we find:
\begin{equation}\label{eq:exprs}
\begin{array}{l|c}
f(x) = 1 - x & \Gamma \\
f_1(x) = f(-x) \implies f_1(x) = 1 + x & \Gamma_1 \\
f_2(x) = -f(-x) \implies f_2(x) = - 1 - x & \Gamma_2 \\
f_3(x) = -f(x) \implies f_3(x) = x - 1 & \Gamma_3 \\
\end{array}
\end{equation}
This completes the first part of the problem.

\subsection{Polynomial decorations and dimensioning}
\label{sec:p1}

The next part of the problem concerns a \emph{dimensioning}\footnote{The
process of calculating the values of the parameters of a system
to reach a specic goal.} task.
The new requirements for tiles can be formalized as follows:
\begin{equation}
\label{eq:fconds2}
\begin{cases}
f^\prime(0) = 0\\
S_{\Gamma} = \gamma \cdot S \implies 
   \int^{1}_{0} f(x) \, \mathrm{d} x = \gamma \cdot S\\
\end{cases}
\end{equation}
Having $\gamma = \frac{55}{100}$ represent the fraction of area inside $\Gamma$
in relation to the whole tile, 
$S_{\Gamma} \in \mathbb{R}$ represent the surface 
inside $\Gamma$ and $S \in \mathbb{R}$ the
surface of the whole tile. The product of these two quantities is the final value 
required by the problem. 
Be carefull not to misunderstand the problem: one single tile contains one $\Gamma$,
the machine will print the other copies on 4 other different tiles. That's why the
percentage $\gamma$ does not refer to the final decoration $\Lambda$, therefore
$S = 1^2 = 1$ and $S \not\eq 2 ^ 2$. 

The problem asks to find a new function to decorate tiles and it must 
satisfy both equations \ref{eq:fconds} and \ref{eq:fconds2}.
We are suggested to use 2\textsuperscript{nd} or 3\textsuperscript{rd} 
order polynomials as the new $f$.

\begin{proposition}[2-degree polynomials don't meet conditions in equations 
\ref{eq:fconds} and \ref{eq:fconds2}]
\label{the:2degpoly}
Let $f:[0,1] \mapsto [0,1]$ be a continuous function in the form: 
$f(x) \doteq ax^2+bx+c$ where $a,b,c \in \mathbb{R}$. 
Then $f$ can never meet conditions as per both
equations \ref{eq:fconds} and \ref{eq:fconds2}.
\begin{proof}
To prove this, we need to plug $f$ inside equations \ref{eq:fconds} and 
\ref{eq:fconds2} and generate a system of equations on parameters $a$, 
$b$ and $c$:
\begin{equation*}
\begin{cases}
f(0)=1 \implies \left. ax^2+bx+c \right|_{x=0} = 1 \implies c = 1 \\
f(1)=0 \implies \left. ax^2+bx+c \right|_{x=1} = 0 \implies a+b+c=0\\
f^\prime(0) = 0 \implies \left. 2ax + b \right|_{x=0} = 0 \implies b = 0 \\
\int^{1}_{0} \left( ax^2+bx+c \right) \, \mathrm{d} x = \gamma S \\
\end{cases}
\end{equation*}
And we must also guarantee that $0 < f(x) < 1, \forall x \in [0,1]. 
$Note that the condition on continuity is observed because $f$ is a polynomial which
is continuous always. 
The first three equations define the values of the three parameters: 
$a=-1$, $b=0$ and $c=1$; which means that $f(x) = -x^2 + 1$. Let's now
verify that this function satisfies the fourth equation in the system:
\begin{align*}
&\int^{1}_{0} \left( -x^2+1 \right) \, \mathrm{d} x = \gamma S
\implies
-\int^{1}_{0} x^2 \, \mathrm{d} x + \int^{1}_{0} \mathrm{d} x = 
        \gamma S \implies \notag\\
&- \left[\frac{x^3}{3}\right]_0^1 + \left[x\right]_0^1 = \gamma S
\implies -\frac{1}{3} + 1 = \frac{55}{100} \cdot 1
\implies \frac{2}{3} = \frac{55}{100}
\end{align*}
The equation above proves that the fourth condition is the system is not met, which
proves the thesis.
$\square$
\end{proof}
\end{proposition}
Let's try with 3\textsuperscript{rd}-degree polynomials:
\begin{proposition}[3-degree polynomials meet conditions in equations 
\ref{eq:fconds} and \ref{eq:fconds2}]
\label{the:3degpoly}
Let $f:[0,1] \mapsto [0,1]$ be a continuous function in the form: 
$f(x) \doteq ax^3+bx^2+cx+d$ where $a,b,c,d \in \mathbb{R}$. 
Then $f$ meets conditions as per both
equations \ref{eq:fconds} and \ref{eq:fconds2}.
\begin{proof}
To prove this, we do as in the proof for proposition \ref{the:2degpoly}:
\begin{equation*}
\begin{cases}
f(0)=1 \implies \left. ax^3+bx^2+cx+d \right|_{x=0} = 1 \implies d = 1 \\
f(1)=0 \implies \left. ax^3+bx^2+cx+d \right|_{x=1} = 0\\
f^\prime(0) = 0 \implies \left. 3ax^2 + 2bx + c \right|_{x=0} = 0 \implies c = 0 \\
\int^{1}_{0} \left( ax^3+bx^2+cx+d \right) \, \mathrm{d} x = \gamma S \\
\end{cases}
\end{equation*}
Here as well, the condition on continuity is observed because $f$ is 
still a polynomial.
Parameters $c$ and $d$ were found, so we can re-write the system:
\begin{equation*}
\begin{cases}
\left. ax^3+bx^2+1 \right|_{x=1} = 0 \implies a+b+1=0\\
\int^{1}_{0} \left( ax^3+bx^2+1 \right) \, \mathrm{d} x = \gamma S \\
\end{cases}
\end{equation*}
Let's work on the last condition:
\begin{align*}
&\int^{1}_{0} \left( ax^3+bx^2+1 \right) \, \mathrm{d} x = \gamma S
\implies
a\int^{1}_{0} x^3 \, \mathrm{d} x + b \int^{1}_{0} x^2 \, \mathrm{d} x \, + \notag\\ 
&+ \int^{1}_{0} \mathrm{d} x= 
        \gamma S \implies 
        a \left[\frac{x^4}{4}\right]_0^1 + b \left[\frac{x^3}{3}\right]_0^1 +
        \left[x\right]_0^1 = \gamma S \implies \notag\\
&\frac{a}{4} + \frac{b}{3} + 1 = \gamma S 
\end{align*}
The final system in $a$ and $b$ can be written now as:
\begin{equation}\label{eq:3degsys}
\begin{cases}
a+b=-1\\
3a + 4b = 12 \gamma S - 12 \\
\end{cases}
\implies
\begin{pmatrix}
1 & 1 \\
3 & 4
\end{pmatrix}
\cdot
\begin{pmatrix}
a \\
b
\end{pmatrix}
=
\begin{pmatrix}
-1 \\
12 \gamma S - 12
\end{pmatrix}
\end{equation}
To verify that 3\textsuperscript{rd}-degree polynomials meet the conditions, we
just need to prove that the system in equation \ref{eq:3degsys} has a solution.
We expressed the system in matricial form to make it evident that this is a 
Cramer system\footnote{A system of $n$ linear equations in $n$ variables.},
which means that two possibilites hold: either the system has one solution only, or 
it has no solutions at all. To see which case we are in, let's calculate the determinant
of the coefficient matrix:
\begin{equation*}
\left|
\begin{matrix}
1 & 1 \\
3 & 4
\end{matrix}
\right|
= 1 \cdot 4 - 3 \cdot 1 = 4 - 3 = 1 \not\eq 0
\end{equation*}
Since the determinant is not zero, the system has one single solution, which means
that there exist values for $a$ and $b$, which together with those of $c$ and $d$ we
found before, determine a function observing equations 
\ref{eq:fconds} and \ref{eq:fconds2}.

At this point, however, our proof is not 100\% complete. We still need to prove,
as part of equation \ref{eq:fconds}, that 
$0 \leq f(x) \leq 1, \forall x \in [0,1]$; we will do this now. We start from one
important consideration: $f$ is a polynomial, it means that it is defined in all
$\mathbb{R}$ with no discontinuities. It means that:
\begin{equation*}
\not\exists x_0 \in \mathbb{R} : \lim_{x \to x_0} f(x) = \infty \wedge
\lim_{x \to \infty} f(x) = \infty
\end{equation*}
The function is a smooth curve which grows to infinity only when $x$ goes to
infinity. Which means that in $x \in [0,1]$, because of Weierstrass'
boundedness theorem, $f(x)$ has local max $M \in \mathbb{R}$ 
and min $m \in \mathbb{R}$ values. However we already know by hypothesis that
$f(0) = 1$ and $f(1) = 0$, and we want $0 \leq f(x) \leq 1$ for $0 \leq x \leq 1$;
therefore we have the following:
\begin{equation*}
m \leq 0 \leq f(x) \leq 1 \leq M, \forall x : 0 \leq x \leq 1
\end{equation*}
%
% Figure
%-----------------------------------------------------------
%
% One column figure
%-----------------------------------------------------------
   \begin{figure}
   \centering
\tikzstyle{smallfig}=[scale=0.25]
   %
\begin{tikzpicture}[style=smallfig]
\draw [fill=lightgray!20, pattern=north west lines, 
   pattern color=red] (0, 4) rectangle (4, 6);
\draw [fill=lightgray!20, pattern=north west lines, 
   pattern color=red] (4, 0) rectangle (6, 4);
\draw [fill=lightgray!20, pattern=north west lines, 
   pattern color=red] (0, -2) rectangle (4, 0);

\draw[thin,->] (-2,0) -- (6,0) node[right] {$\hat{x}$};
\draw[thin,->] (0,-2) -- (0,6) node[above] {$\hat{y}$};

\draw[dashed,thin,-] (-2,4) -- (6,4);
\draw[dashed,thin,-] (4,-2) -- (4,6);
\fill (0,4) circle (5pt) node[below left] {$1$};
\fill (4,0) circle (5pt) node[below right] {$1$};
\end{tikzpicture}
\quad
\begin{tikzpicture}[style=smallfig]
\draw[thin,->] (-2,0) -- (6,0) node[right] {$\hat{x}$};
\draw[thin,->] (0,-2) -- (0,6) node[above] {$\hat{y}$};

\draw[dashed,thin,-] (-2,4) -- (6,4);
\fill (0,4) circle (5pt) node[below left] {$1$};
\fill (4,0) circle (5pt) node[below left] {$1$};

\draw[thick,->, red] (-1,4) -- (1,4) node[above right] {$f^\prime(0)=0$};
\draw[thick,->, red] (3,1) node[above right] {$f(1)=0$} -- (5,-1);
\end{tikzpicture}
\quad
\begin{tikzpicture}[style=smallfig]
\draw[thin,->] (-2,0) -- (6,0) node[right] {$\hat{x}$};
\draw[thin,->] (0,-2) -- (0,6) node[above] {$\hat{y}$};

\fill (0,4) circle (5pt) node[below left] {$1$};
\fill (4,0) circle (5pt) node[below left] {$1$};

\begin{scope}[thick, decoration={
    markings,
    mark=at position 0.5 with {\arrow{>}}}
    ] 

    \draw[postaction={decorate}] (0,4) to[out=0,in=100] (4,0);
    \draw[postaction={decorate}] (0,4) -- (4,0);
\end{scope}
\end{tikzpicture}
   %
   \caption{Showing the how the constraints on $f$ define its shape
   in $[0,1] \times [0,1]$. The first picture shows the regione where $f$
   can lay, the middle one shows tangent in $x=0$ and the condition
   of crossing in $(1,0)$. The rightmost picture shows possible
   shapes of $f$: note how the concavity cannot be pointing upwards.}
   \label{fig:fshape}
   \end{figure}
%-----------------------------------------------------------
%
%-----------------------------------------------------------
%
This means that $m=0$ and $M = 1$. So in $x\in[0,1]$, there can only be one
relative min and one relative max. We need to prove this. Together
with the fact that $f$ is continuous, we basically need to prove that $f$ is
monotonically decreasing inside $[0,1]$ like figure \ref{fig:fshape} shows. To
prove this, we must study $f$'s monotonicity: the most classic way to do this
is by considering its first-order derivative $f^\prime$ and verify that
$f^\prime(x) < 0, \forall x \in ]0,1[$:
\begin{align*}
&f^\prime(x) > 0 \implies = \frac{d}{dx}(ax^3+bx^2+1) > 0 \implies \notag\\
&3ax^2+2bx > 0 \implies (3ax + 2b) x > 0
\end{align*}
By studying the sign of the derivative, we can see its monotonicity. Of course we will
focus on $x\in]0,1[$ and we are going to check that $f < 0$ in that interval. The above
inequality has been rearranged in the product of two expressions whose sign can be
studied separately: $3ax + 2b > 0 \implies x > -\frac{2}{3}\frac{b}{a}$ and $x > 0$.
%
% Figure
%-----------------------------------------------------------
%
% One column figure
%-----------------------------------------------------------
   \begin{figure}
   \centering
\tikzstyle{smallfig}=[scale=0.25]
   %
\begin{tikzpicture}[style=smallfig]
\tikzmath{
	\X1 = 0; \X2 = 8;
	\x1 = 2; \x2 = 2*\x1; \x3 = 3*\x1; 
	\xv1 = 0.5*\x1; 
	\xv2 = \x1 + 0.5*(\x2 - \x1); 
	\xv3 = \x3;
	\y1 = 1.5; \y2 = 2*\y1; \y3 = 4*\y1;
	\ya = 5;
}

\draw[thin,->] (\X1,0) -- (\X2,0) node[right] {$x$};

\draw[dotted,thin,-] (\x1,0) -- (\x1,\y3);
\draw[dotted,thin,-] (\x2,0) -- (\x2,\y3);

\draw[dashed,thin,-] (0,\y2) -- (\x1,\y2);
\draw[thin,-] (\x1,\y2) -- (\X2,\y2) node[right] {\tiny $E_1$};
\draw[thin,-] (\x1,0) -- (\x1,\y2);

\draw[dashed,thin,-] (\X1,\y1) -- (\x2,\y1);
\draw[thin,-] (\x2,\y1) -- (\X2,\y1) node[right] {\tiny $E_2$};
\draw[thin,-] (\x2,0) -- (\x2,\y1);

\node[] at (\xv1,0.5*\y1) {\tiny $-$};
\node[] at (\xv2,0.5*\y1) {\tiny $-$};
\node[] at (\xv3,0.5*\y1) {\tiny $+$};

\node[] at (\xv1,1.5*\y1) {\tiny $-$};
\node[] at (\xv2,1.5*\y1) {\tiny $+$};
\node[] at (\xv3,1.5*\y1) {\tiny $+$};

\node[blue] at (\xv1,2.5*\y1) {\tiny $+$};
\node[red] at (\xv2,2.5*\y1) {\tiny $-$};
\node[blue] at (\xv3,2.5*\y1) {\tiny $+$};

\node[blue] at (\xv1,\ya) {$\nearrow$};
\node[red] at (\xv2,\ya) {$\searrow$};
\node[blue] at (\xv3,\ya) {$\nearrow$};

\fill (\x1,0) circle (5pt) node[below] {$\beta$};
\fill (\x2,0) circle (5pt) node[below] {$0$};
\fill (\x3,0) circle (5pt) node[below] {$1$};
\end{tikzpicture}
\quad
\begin{tikzpicture}[style=smallfig]
\tikzmath{
	\X1 = 0; \X2 = 8;
	\x1 = 2; \x2 = 2*\x1; \x3 = 3*\x1; 
	\xv1 = 0.5*\x1; 
	\xv2 = \x1 + 0.5*(\x2 - \x1); 
	\xv3 = \x3;
	\y1 = 1.5; \y2 = 2*\y1; \y3 = 4*\y1;
	\ya = 5;
}

\draw[thin,->] (\X1,0) -- (\X2,0) node[right] {$x$};

\draw[dotted,thin,-] (\x1,0) -- (\x1,\y3);
\draw[dotted,thin,-] (\x2,0) -- (\x2,\y3);

\draw[dashed,thin,-] (\X1,\y1) -- (\x1,\y1);
\draw[thin,-] (\x1,\y1) -- (\X2,\y1) node[right] {\tiny $E_1$};
\draw[thin,-] (\x1,0) -- (\x1,\y1);

\draw[dashed,thin,-] (0,\y2) -- (\x2,\y2);
\draw[thin,-] (\x2,\y2) -- (\X2,\y2) node[right] {\tiny $E_2$};
\draw[thin,-] (\x2,0) -- (\x2,\y2);

\node[] at (\xv1,0.5*\y1) {\tiny $-$};
\node[] at (\xv2,0.5*\y1) {\tiny $+$};
\node[] at (\xv3,0.5*\y1) {\tiny $+$};

\node[] at (\xv1,1.5*\y1) {\tiny $-$};
\node[] at (\xv2,1.5*\y1) {\tiny $-$};
\node[] at (\xv3,1.5*\y1) {\tiny $+$};

\node[blue] at (\xv1,2.5*\y1) {\tiny $+$};
\node[red] at (\xv2,2.5*\y1) {\tiny $-$};
\node[blue] at (\xv3,2.5*\y1) {\tiny $+$};

\node[blue] at (\xv1,\ya) {$\nearrow$};
\node[red] at (\xv2,\ya) {$\searrow$};
\node[blue] at (\xv3,\ya) {$\nearrow$};

\fill (\x1,0) circle (5pt) node[below] {$0$};
\fill (\x2,0) circle (5pt) node[below] {$\beta$};
\fill (\x3,0) circle (5pt) node[below] {$1$};
\end{tikzpicture}
\quad
\begin{tikzpicture}[style=smallfig]
\tikzmath{
	\X1 = 0; \X2 = 8;
	\x1 = 2; \x2 = 2*\x1; \x3 = 3*\x1; 
	\xv1 = 0.5*\x1; 
	\xv2 = \x2; 
	\xv3 = \x3 + 0.5*(\X2 - \x3);
	\y1 = 1.5; \y2 = 2*\y1; \y3 = 4*\y1;
	\ya = 5;
}

\draw[thin,->] (\X1,0) -- (\X2,0) node[right] {$x$};

\draw[dotted,thin,-] (\x1,0) -- (\x1,\y3);
\draw[dotted,thin,-] (\x3,0) -- (\x3,\y3);

\draw[dashed,thin,-] (\X1,\y1) -- (\x1,\y1);
\draw[thin,-] (\x1,\y1) -- (\X2,\y1) node[right] {\tiny $E_1$};
\draw[thin,-] (\x1,0) -- (\x1,\y1);

\draw[dashed,thin,-] (0,\y2) -- (\x3,\y2);
\draw[thin,-] (\x3,\y2) -- (\X2,\y2) node[right] {\tiny $E_2$};
\draw[thin,-] (\x3,0) -- (\x3,\y2);

\node[] at (\xv1,0.5*\y1) {\tiny $-$};
\node[] at (\xv2,0.5*\y1) {\tiny $-$};
\node[] at (\xv3,0.5*\y1) {\tiny $+$};

\node[] at (\xv1,1.5*\y1) {\tiny $-$};
\node[] at (\xv2,1.5*\y1) {\tiny $+$};
\node[] at (\xv3,1.5*\y1) {\tiny $+$};

\node[red] at (\xv1,2.5*\y1) {\tiny $-$};
\node[red] at (\xv2,2.5*\y1) {\tiny $-$};
\node[blue] at (\xv3,2.5*\y1) {\tiny $+$};

\node[red] at (\xv1,\ya) {$\searrow$};
\node[red] at (\xv2,\ya) {$\searrow$};
\node[blue] at (\xv3,\ya) {$\nearrow$};

\fill (\x1,0) circle (5pt) node[below] {$0$};
\fill (\x2,0) circle (5pt) node[below] {$1$};
\fill (\x3,0) circle (5pt) node[below] {$\beta$};
\end{tikzpicture}
   %
   \caption{Showing the how the constraints on $f$ define its shape
   in $[0,1] \times [0,1]$. The first picture shows the regione where $f$
   can lay, the middle one shows tangent in $x=0$ and the condition
   of crossing in $(1,0)$. The rightmost picture shows possible
   shapes of $f$: note how the concavity cannot be pointing upwards.}
   \label{fig:fshape}
   \end{figure}
%-----------------------------------------------------------
%
%-----------------------------------------------------------
%
The results of the inequality depend on $\beta = -\frac{2}{3}\frac{b}{a}$.
$\square$
\end{proof}
\end{proposition}
We are asked to find the values of $a$, $b$, $c$ and $d$ though. Since $c=0$ and
$d=1$, we need to calculate $a$ and $b$ from equation \ref{eq:3degsys}.
We simple rearrange the matricial equation multiplying both members by the
inverse of the coefficient matrix (on the left side of course):
\begin{align*}
&
\begin{pmatrix}
1 & 1 \\
3 & 4
\end{pmatrix}^{-1}
\cdot
\begin{pmatrix}
1 & 1 \\
3 & 4
\end{pmatrix}
\cdot
\begin{pmatrix}
a \\
b
\end{pmatrix}
=
\begin{pmatrix}
1 & 1 \\
3 & 4
\end{pmatrix}^{-1}
\cdot
\begin{pmatrix}
-1 \\
12 \gamma S - 12
\end{pmatrix}
\implies \notag\\
&\begin{pmatrix}
a \\
b
\end{pmatrix}
=
\begin{pmatrix}
1 & 1 \\
3 & 4
\end{pmatrix}^{-1}
\cdot
\begin{pmatrix}
-1 \\
12 \gamma S - 12
\end{pmatrix}
\end{align*}
Let's calculate the inverse of the coefficient matrix:
\begin{align*}
&
\begin{pmatrix}
1 & 1 \\
3 & 4
\end{pmatrix}^{-1}
=
\begin{bmatrix}
(-1)^{1+1}C_{1,1} & (-1)^{1+2}C_{1,2} \\
(-1)^{2+1}C_{2,1} & (-1)^{2+2}C_{2,2}
\end{bmatrix}^\text{T}
= \notag\\
&=\begin{bmatrix}
(-1)^{1+1} \cdot 4 & (-1)^{1+2} \cdot 3 \\
(-1)^{2+1} \cdot 1 & (-1)^{2+2} \cdot 1
\end{bmatrix}^\text{T}
=
\begin{pmatrix}
4 & -3 \\
-1 & 1
\end{pmatrix}^\text{T}
=
\begin{pmatrix}
4 & -1 \\
-3 & 1
\end{pmatrix}
\end{align*}
So, we can now calculate $a$ and $b$:
\begin{equation*}
\begin{pmatrix}
a \\
b
\end{pmatrix}
=
\begin{pmatrix}
4 & -1 \\
-3 & 1
\end{pmatrix}
\cdot
\begin{pmatrix}
-1 \\
12 \gamma S - 12
\end{pmatrix}
=
\begin{pmatrix}
-4 +12 -12 \gamma S \\
3 - 12 + 12 \gamma S
\end{pmatrix}
\end{equation*}
Considering that $12 \gamma S = 12 \cdot \frac{55}{100} \cdot 1 = \frac{66}{10}$,
we have that $a = 8 - \frac{66}{10} = \frac{7}{5}$, and 
$b = -9 + \frac{66}{10} = -\frac{12}{5}$.

   is a thermodynamical quantity which is of order $1$ and equal to $1$
   for nonreacting mixtures of classical perfect gases. The physical
   meaning of $ \sigma_0 $ and $K$ is clearly visible in the equations
   above. $\sigma_0$ represents a frequency of the order one per
   free-fall time. $K$ is proportional to the ratio of the free-fall
   time and the cooling time. Substituting into Baker's criteria, using
   thermodynamic identities and definitions of thermodynamic quantities,
   \begin{displaymath}
      \Gamma_1      = \left( \frac{ \partial \ln P}{ \partial\ln \rho}
                           \right)_{S}    \, , \;
      \chi^{}_\rho  = \left( \frac{ \partial \ln P}{ \partial\ln \rho}
                           \right)_{T}    \, , \;
      \kappa^{}_{P} = \left( \frac{ \partial \ln \kappa}{ \partial\ln P}
                           \right)_{T}
   \end{displaymath}
   \begin{displaymath}
      \nabla_{\mathrm{ad}} = \left( \frac{ \partial \ln T}
                             { \partial\ln P} \right)_{S} \, , \;
      \chi^{}_T       = \left( \frac{ \partial \ln P}
                             { \partial\ln T} \right)_{\rho} \, , \;
      \kappa^{}_{T}   = \left( \frac{ \partial \ln \kappa}
                             { \partial\ln T} \right)_{T}
   \end{displaymath}
   one obtains, after some pages of algebra, the conditions for
   \emph{stability\/} given
   below:
   \begin{eqnarray}
      \frac{\pi^2}{8} \frac{1}{\tau_{\mathrm{ff}}^2}
                ( 3 \Gamma_1 - 4 )
         & > & 0 \label{ZSDynSta} \\
      \frac{\pi^2}{\tau_{\mathrm{co}}
                   \tau_{\mathrm{ff}}^2}
                   \Gamma_1 \nabla_{\mathrm{ad}}
                   \left[ \frac{ 1- 3/4 \chi^{}_\rho }{ \chi^{}_T }
                          ( \kappa^{}_T - 4 )
                        + \kappa^{}_P + 1
                   \right]
        & > & 0 \label{ZSSecSta} \\
     \frac{\pi^2}{4} \frac{3}{\tau_{ \mathrm{co} }
                              \tau_{ \mathrm{ff} }^2
                             }
         \Gamma_1^2 \, \nabla_{\mathrm{ad}} \left[
                                   4 \nabla_{\mathrm{ad}}
                                   - ( \nabla_{\mathrm{ad}} \kappa^{}_T
                                     + \kappa^{}_P
                                     )
                                   - \frac{4}{3 \Gamma_1}
                                \right]
        & > & 0   \label{ZSVibSta}
   \end{eqnarray}
%
   For a physical discussion of the stability criteria see Baker
   (\cite{baker}) or Cox (\cite{cox}).

   We observe that these criteria for dynamical, secular and
   vibrational stability, respectively, can be factorized into
   \begin{enumerate}
      \item a factor containing local timescales only,
      \item a factor containing only constitutive relations and
         their derivatives.
   \end{enumerate}
   The first factors, depending on only timescales, are positive
   by definition. The signs of the left hand sides of the
   inequalities~(\ref{ZSDynSta}), (\ref{ZSSecSta}) and (\ref{ZSVibSta})
   therefore depend exclusively on the second factors containing
   the constitutive relations. 

%__________________________________________________________________
\section{Second problem}

   Since they depend only
   on state variables, the stability criteria themselves are \emph{
   functions of the thermodynamic state in the local zone}. The
   one-zone stability can therefore be determined
   from a simple equation of state, given for example, as a function
   of density and
   temperature. Once the microphysics, i.e.\ the thermodynamics
   and opacities (see Table~\ref{KapSou}), are specified (in practice
   by specifying a chemical composition) the one-zone stability can
   be inferred if the thermodynamic state is specified.
   The zone -- or in
   other words the layer -- will be stable or unstable in
   whatever object it is imbedded as long as it satisfies the
   one-zone-model assumptions. Only the specific growth rates
   (depending upon the time scales) will be different for layers
   in different objects.

%__________________________________________________ One column table
   \begin{table}
      \caption[]{Opacity sources.}
         \label{KapSou}
     $$ 
         \begin{array}{p{0.5\linewidth}l}
            \hline
            \noalign{\smallskip}
            Source      &  T / {[\mathrm{K}]} \\
            \noalign{\smallskip}
            \hline
            \noalign{\smallskip}
            Yorke 1979, Yorke 1980a & \leq 1700^{\mathrm{a}}     \\
%           Yorke 1979, Yorke 1980a & \leq 1700             \\
            Kr\"ugel 1971           & 1700 \leq T \leq 5000 \\
            Cox \& Stewart 1969     & 5000 \leq             \\
            \noalign{\smallskip}
            \hline
         \end{array}
     $$ 
   \end{table}
%
   We will now write down the sign (and therefore stability)
   determining parts of the left-hand sides of the inequalities
   (\ref{ZSDynSta}), (\ref{ZSSecSta}) and (\ref{ZSVibSta}) and thereby
   obtain \emph{stability equations of state}.

   The sign determining part of inequality~(\ref{ZSDynSta}) is
   $3\Gamma_1 - 4$ and it reduces to the
   criterion for dynamical stability
   \begin{equation}
     \Gamma_1 > \frac{4}{3}\,\cdot
   \end{equation}
   Stability of the thermodynamical equilibrium demands
   \begin{equation}
      \chi^{}_\rho > 0, \;\;  c_v > 0\, ,
   \end{equation}
   and
   \begin{equation}
      \chi^{}_T > 0
   \end{equation}
   holds for a wide range of physical situations.
   With
   \begin{eqnarray}
      \Gamma_3 - 1 = \frac{P}{\rho T} \frac{\chi^{}_T}{c_v}&>&0\\
      \Gamma_1     = \chi_\rho^{} + \chi_T^{} (\Gamma_3 -1)&>&0\\
      \nabla_{\mathrm{ad}}  = \frac{\Gamma_3 - 1}{\Gamma_1}         &>&0
   \end{eqnarray}
   we find the sign determining terms in inequalities~(\ref{ZSSecSta})
   and (\ref{ZSVibSta}) respectively and obtain the following form
   of the criteria for dynamical, secular and vibrational
   \emph{stability}, respectively:
   \begin{eqnarray}
      3 \Gamma_1 - 4 =: S_{\mathrm{dyn}}      > & 0 & \label{DynSta}  \\
%
      \frac{ 1- 3/4 \chi^{}_\rho }{ \chi^{}_T } ( \kappa^{}_T - 4 )
         + \kappa^{}_P + 1 =: S_{\mathrm{sec}} > & 0 & \label{SecSta} \\
%
      4 \nabla_{\mathrm{ad}} - (\nabla_{\mathrm{ad}} \kappa^{}_T
                             + \kappa^{}_P)
                             - \frac{4}{3 \Gamma_1} =: S_{\mathrm{vib}}
                                      > & 0\,.& \label{VibSta}
   \end{eqnarray}
   The constitutive relations are to be evaluated for the
   unperturbed.

%__________________________________________________________________
\section{Questionnaire}

\subsection{First question}
   Thermodynamic state (say $(\rho_0, T_0)$) of the zone.
   We see that the one-zone stability of the layer depends only on
   the constitutive relations $\Gamma_1$,
   $\nabla_{\mathrm{ad}}$, $\chi_T^{},\,\chi_\rho^{}$,
   $\kappa_P^{},\,\kappa_T^{}$.

\subsection{Second question}
   Thermodynamic state (say $(\rho_0, T_0)$) of the zone.
   We see that the one-zone stability of the layer depends only on
   the constitutive relations $\Gamma_1$,
   $\nabla_{\mathrm{ad}}$, $\chi_T^{},\,\chi_\rho^{}$,
   $\kappa_P^{},\,\kappa_T^{}$.

\subsection{Third question}
   These depend only on the unperturbed
   thermodynamical state of the layer. Therefore the above relations
   define the one-zone-stability equations of state
   $S_{\mathrm{dyn}},\,S_{\mathrm{sec}}$
   and $S_{\mathrm{vib}}$. See Fig.~\ref{FigVibStab} for a picture of
   $S_{\mathrm{vib}}$. Regions of secular instability are
   listed in Table~1.

%
% Two column figure (place early!)
%-----------------------------------------------------------
   \begin{figure*}
   \centering
   %%%\includegraphics{empty.eps}
   %%%\includegraphics{empty.eps}
   %%%\includegraphics{empty.eps}
   \caption{Adiabatic exponent $\Gamma_1$.
               $\Gamma_1$ is plotted as a function of
               $\lg$ internal energy $\mathrm{[erg\,g^{-1}]}$ and $\lg$
               density $\mathrm{[g\,cm^{-3}]}$.}
              \label{FigGam}%
    \end{figure*}
%-----------------------------------------------------------
%
%______________________________________________________________

\section{Conclusions}

   \begin{enumerate}
      \item The conditions for the stability of static, radiative
         layers in gas spheres, as described by Baker's (\cite{baker})
         standard one-zone model, can be expressed as stability
         equations of state. These stability equations of state depend
         only on the local thermodynamic state of the layer.
      \item If the constitutive relations -- equations of state and
         Rosseland mean opacities -- are specified, the stability
         equations of state can be evaluated without specifying
         properties of the layer.
      \item For solar composition gas the $\kappa$-mechanism is
         working in the regions of the ice and dust features
         in the opacities, the $\mathrm{H}_2$ dissociation and the
         combined H, first He ionization zone, as
         indicated by vibrational instability. These regions
         of instability are much larger in extent and degree of
         instability than the second He ionization zone
         that drives the Cephe{\"\i}d pulsations.
   \end{enumerate}

\begin{acknowledgements}
      Part of this work was supported by the German
      \emph{Deut\-sche For\-schungs\-ge\-mein\-schaft, DFG\/} project
      number Ts~17/2--1.
\end{acknowledgements}


%-------------------------------------------------------------------

\begin{thebibliography}{}

  \bibitem[First Problem]{exam1} MIUR 2018,
      in Esame di Stato di Istruzione Secondaria Superiore anno 2018,
      I043 sessione ordinaria,
      Ministero dell'Istruzione, dell'Universit\a' e della Ricerca - Problema 1

  \bibitem[First Problem]{exam2} MIUR 2018,
      in Esame di Stato di Istruzione Secondaria Superiore anno 2018,
      I043 sessione ordinaria,
      Ministero dell'Istruzione, dell'Universit\a' e della Ricerca - Problema 2

  \bibitem[First Problem]{examQ} MIUR 2018,
      in Esame di Stato di Istruzione Secondaria Superiore anno 2018,
      I043 sessione ordinaria,
      Ministero dell'Istruzione, dell'Universit\a' e della Ricerca - Questionario

  \bibitem[1966]{baker} Baker, N. 1966,
      in Stellar Evolution,
      ed.\ R. F. Stein,\& A. G. W. Cameron
      (Plenum, New York) 333

   \bibitem[1988]{balluch} Balluch, M. 1988,
      A\&A, 200, 58

   \bibitem[1980]{cox} Cox, J. P. 1980,
      Theory of Stellar Pulsation
      (Princeton University Press, Princeton) 165

   \bibitem[1969]{cox69} Cox, A. N.,\& Stewart, J. N. 1969,
      Academia Nauk, Scientific Information 15, 1

   \bibitem[1980]{mizuno} Mizuno H. 1980,
      Prog. Theor. Phys., 64, 544
   
   \bibitem[1987]{tscharnuter} Tscharnuter W. M. 1987,
      A\&A, 188, 55
  
   \bibitem[1992]{terlevich} Terlevich, R. 1992, in ASP Conf. Ser. 31, 
      Relationships between Active Galactic Nuclei and Starburst Galaxies, 
      ed. A. V. Filippenko, 13

   \bibitem[1980a]{yorke80a} Yorke, H. W. 1980a,
      A\&A, 86, 286

   \bibitem[1997]{zheng} Zheng, W., Davidsen, A. F., Tytler, D. \& Kriss, G. A.
      1997, preprint
\end{thebibliography}

\end{document}
